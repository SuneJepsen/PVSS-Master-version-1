\section{Functional requirements}
\begin{itemize}
    \item \textit{Voter Privacy}
        No one should be able to link a vote back to the specific voter, and only the voter should
        know his vote. These requirements shall hold during and after the election.  
    
    \item \textit{Eligibility}    
        Only Eligible and registered voters can vote. 
    
    \item \textit{Uniqueness}
        Only one vote per registered voter should be counted.
    
    \item \textit{Fairness}
        None should be able to gain any knowledge of the outcome of the election, before the ending. This is to prevent voters of voting accordingly to any leaked information. 
    
    \item \textit{Uncoercibility}
        Nobody should be able to extract the value of a vote. This is to prevent anybody from compelling a voter by force, intimidation, or authority to cast a vote in a specific way. 
    
    \item \textit{Receipt-freeness} 
        The voting system should not produce a receipt that reveals any information about the casted vote. This is to prevent a vote from trading his vote. 
    
    \item \textit{Accuracy} 
        The final tally should be correctly computed from valid casted votes. It should not be
        possible to manipulate the final tally without being detected. 
    
    \item \textit{Universal Verifiability}
        It should be possible for any participants and observers to validate individual votes as well as the final tally of the election. 
    
    \item \textit{Individual Verifiability}    
        Every registered voter should be able to verify that his vote is counted correctly. 
    
\end{itemize}
Use cases: Only registers user are allowed to votes one time