In this Master thesis we will study the public verifiable secret sharing protocol and how it can be used in an electronic voting application based on the work from \cite{Schoenmakers1999}. Based on this knowledge we will design an implement a web based electronic voting application.\\

\noindent
Our work with this protocol leads to the following main topics which should cover our objective about secret sharing, multiparty computation, public verifiable secret sharing protocol and our implementation of an electronic voting application.

\begin{enumerate}
    \item Voting
    \item Mathematical understanding
    \item Multiparty computation
    \item Electronic voting protocol
    \item Designing the application
    \item The application
    \item Reflection
\end{enumerate}

\noindent
We start to describe the concept of electronic voting and the challenges with the different types of electronic voting applications. We will use other studies and their demands for concrete security requirements, which we can include in our consideration for our electronic voting application.  \\

\noindent
To understand the public verifiable secret sharing protocol one need some basic mathematical understanding and some knowledge about cryptographic tools. Modular arithmetic and group theory will be key elements in understanding how the protocol works. Regarding to the cryptographic tools we will present the discrete logarithm problem which is the security primitive for this protocol.\\

\noindent
The public verifiable secret sharing protocol is an extension of multiparty computation protocol. Since the multiparty computation protocol uses secret sharing we need a basic understanding on these two subjects. Multiparty computation is basically about allowing parties to compute some function on some private inputs, in such a way that they learn the result but not the inputs from the other parties. Regarding secret sharing is about hiding information in a random polynomial. Based on this polynomial, parties can create shares based on evaluation in the polynomial. If enough parties then collect their shares together they will be able to recover the secret.  We will present a simple secret sharing example which illustrate how a secret can be distributed and reconstructed. In addition to these properties the public verifiable secret sharing protocol gives us the ability to publicly verify the validity of the shares among the parties involved in this protocol. This means that the protocol is secure against malicious parties which try to send votes which they are not supposed to do. \\

\noindent
The electronic voting protocol is splitted into two part a basic and a more depth description of the protocol. The first part is intended to a developer how can implement a the parts of the protocol. The second part is a more in depth part on the verifying part of the protocol and the mathematical justification of why it works. \\   


\noindent
Designing the application is about our architectural strategies for our application based on the knowledge from literature from \cite{Bass} and \cite{Baerbak10}. We took the security requirements from the previous as the demands for our application. To extract the architectural demands for our application we will use Quality attribute scenarios which is a way of defining a clear architectural measurable demand. In addition for these demands we support this with diagrams which show the influence on the architecture of the application. The process of deriving these demands is done through a Quality attribute workshop. Since we are in  limited  time we onlys used the structure of deriving the Quality attributes scenarios. The structure of the workshop helped us prioritize among a long list of demands, which gave us the most important scenarios which we had to implement.\\


\noindent
The application is about giving an overview how our application is implemented with focus on the architectural elements. This part also contain our perspective of how well we our application obeyed the security requirements. \\


\noindent
The reflection summarizes the most important reflections on our results from the theoretical and the practical parts of this thesis. 



  


 