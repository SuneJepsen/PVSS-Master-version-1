In the PVSS protocol there will be $m$-voters where each voter votes $0$ or $1$. The result of the votation is the sum of the votes. The votes are going to be counted by $n$-talliers where no tallier knows the individual votes, even if $t-1$ talliers are corrupted by the same adversary. There will be a bullutin board where voters/talliers put their public information. This allows the correctness of the voter and tallier to be publicly verified.\\

\noindent
The PVSS protocol is a protocol which can handle a large set of secret data but it can also handle a small set data. This means that one of its applications is a  electronic voting system where the secret may differ from 0 or 1. The PVSS protocol differs from standard MPC protocol in which the shares are made public and verifiable.\\\\
This section will start by explaining our requirements for our implementation and after that we will explain the protocol. The main part will be the Ballot casting where the voter distribute their shares and the Tallying where the tallier computes and reconstructs the votes. The last part will go in depth with the mathematical proofs. \\

\noindent

\begin{infobox}[Short overview of the protocol]
\begin{enumerate}
\item Each voter signs it name on the bulletin board
\item Each tallier generates a private and a public key
\item Each tallier registries their public keys on the bulletin board
\item Initialization
\begin{enumerate}
\item The bulletin board generates and publish $g$ and $G$ and a securetity parameter $t$
\end{enumerate}
\item Ballot casting
\begin{enumerate}
\item Each voter votes $1$ or $0$
\item Each voter generates a random secret
\item Each voter creates shares of the secret to each tallier and encrypts it with the corresponding public key of the tallier. Each voter supply this secret share with evidence of its consistency with a $DLEQ$ proof
\item Each voter supply evidence for a valid vote with $PROOF_U$
\end{enumerate}
\item Tallying
\begin{enumerate}
\item At least $t-1$ tallier accumulates and decrypts their shares
\item One authority completes the final computation of the total votes
\end{enumerate}
\end{enumerate}
\end{infobox}

\noindent
In the following we will describe the central parts of the protocol. We will use terms as Bulletin board which is keen of public board where all the public values gets posted. Tallying is in this context those who counts the votes.

%--------------------------------------------------------------------------
\section{Initialization}
%--------------------------------------------------------------------------
\parahead{Strong prime} In our implementation we will pick a prime, \begin{math}q\end{math}, so we avoid doing the gcd computation. The protocol states that we have to compute in a group of order q. This means that when we are doing operations in the exponent this property should be satisfied \begin{math}g^q=1\end{math} where \begin{math}q\end{math} is prime. If we are doing \begin{math}mod \ q \end{math} in the exponent we have \begin{math}g^q=g^0\end{math}. The reason for doing operation in the exponent \begin{math}mod \ q\end{math} is because we are using Sharmir secret sharing which require a finite field.\\\\
One can see that if \begin{math}q\end{math} is a prime say \begin{math}q=5\end{math}, then \begin{math}2^5 \ mod \ 5 = 32 \ mod \ 5 = 2\end{math}. For this to be true, we take the square of numbers modulo a prime in this form \begin{math}2q+1\end{math}. This is also called a strong prime. By using this mathematical structure this property holds. We can choose \begin{math}b=a^2\end{math}. Then we see the property holds \begin{math}b^{2q} = 1 \ mod \ 2q+1\end{math}. By example it is clear that \begin{math}(2^2)^5 \ mod \ 11 = 1024 \ mod \ 11 = 1\end{math}. Fermat little theorem states that \begin{math}b^{q-1} \ mod \ q = 1\end{math} where \begin{math}q\end{math} is prime. So we know if we pick our \begin{math}q\end{math} and \begin{math}b\end{math} (as a square)  in this form \begin{math}(b^{2})^{q+1-1} \ mod \ 2q+1 =1 =  b^{2q} \ mod \ 2q+1 =1\end{math} the property holds.

\noindent
\begin{infobox}[The bulleting board generates all system parameters]
Public elements:\\
\begin{math}f \in_R \{1,...,2q\} \rightarrow g= f^2 \ mod\ 2q+1\end{math}\\
\begin{math}F \in_R \{1,...,2q\}\rightarrow G= F^2 \ mod\ 2q+1 \wedge G\neq1\end{math}\\
Security parameter:\\
\begin{math} t \in_R \Z_q^* = \{1,2,3,...,q-1\} \end{math}
\end{infobox}


\noindent
There are \textit{m} voters and \textit{n} talliers. For efficiency we will limit the computation of the votes to a finite number of talliers. 

\noindent
\begin{infobox}[The tallier generates a private and a public keys]
Private key: \begin{math}x_{i} \in_R \Z_q^* = \{1,2,3,...,q-1\}\end{math}\\
Public key: \begin{math}y_i=G^{x_i} ,\ i=\{1,2,3,...., n \}  \end{math}
\end{infobox}


\noindent
Every voter either votes "no" or "yes" corresponding to 0 or 1. The voter select a random secret value \begin{math}s \end{math}. The PVSS protocol is then used to distribute shares which contain a combination of the random secret and the vote. Every voter will construct a random polynomium at degree $t$ and then evaluate the shares  of each of the talliers.


\noindent
\begin{infobox}[The voter creates a random polynomial of degree at most $t$ and computes the shares]
Vote: \begin{math}v\in\{0,1\} \end{math}\\
Random value: \begin{math} s\in_R \Z_q \end{math}\\ 
Shamir Secret Sharing: \begin{math} p(x)=s+\alpha_1x^1+\alpha_2x^2+,...,+\alpha_{t-1}x^{t-1}, \ \alpha_j\in_R \Z_q  \end{math}\\
Secret Shares: \begin{math} p(0)=s,\ p(1),\ p(2),...,\ p(n) \end{math}
\end{infobox}


%--------------------------------------------------------------------------
\section{Ballot casting}
%--------------------------------------------------------------------------
The  Ballot casting consists of \textit{distribution of the shares} and \textit{verification of the shares}. First the voter votes and then creates a random value. Two proofs are included by the voter a $DLEQ$ and a $PROOF_U$.\\

\parahead{Shamir Secret Sharing} The \begin{math}p(i)\end{math} is the share where we use Shamir Secret Sharing. The \begin{math}\alpha_j\end{math} is all the coefficients. The shares are going to be encrypted using the talliers public key $y_i^{p(i)}$. \\


\begin{infobox}[The voter distributes the encrypted share with a $PROOF_U$ and a $DLEQ$ proof]
The following gets publish to the bulleting board\\
\begin{math}Y_i=y_i^{p(i)} \end{math}, \begin{math}1\leq i\leq n\end{math} (Encryption of the share) \\ 
\begin{math}C_j=g^{\alpha_j},\ j=\{0,1,2,3,....,t-1 \} \end{math} \\ 
\begin{math}X_i=\prod\limits_{j=0}^{t-1} C_j^{i^j} =g^{p(i)}\end{math}, \begin{math}1\leq i\leq n\end{math}\\
\begin{math}U=G^{s+v} \end{math}\\
\begin{math}PROOF_U \end{math} and DLEQ
\end{infobox}

\parahead{DLEQ} With DLEQ we prove that the exponent are equal \begin{math}X_i=g^{p(i)}  \land Y_i=y_i^{p(i)} \end{math} without revealing \begin{math}{p(i)} \end{math}. Note that there has been no communcation between the the player for now. We will present an interactive and a non-interactive proof of the DLEQ between the voter and the verifier.\\

\parahead{\begin{math} PROOF_U \end{math}} With \begin{math} PROOF_U \end{math} the voter proofs "consistency" and that the vote is either 1 or 0.\\

\noindent
Note that the contruction of $Y_i$ is a hard problem to solve as we discussed in the discrete logarithm section. 

\parahead{Example} For clarification an example on \begin{math}X_i\end{math} calculations to tallier 1, tallier 2 and tallier 3.
\begin{center}
Tallier 1: \begin{math}X_1=\prod\limits_{j=0}^{t-1} C_j^{i^j} = C_0^{1^0}  \cdot  C_1^{1^1}  \cdot  C_2^{1^2}\end{math}\\
Tallier 2: \begin{math}X_2=\prod\limits_{j=0}^{t-1} C_j^{i^j} = C_0^{2^0}  \cdot  C_1^{2^1}  \cdot  C_2^{2^2}\end{math}\\
Tallier 3: \begin{math}X_3=\prod\limits_{j=0}^{t-1} C_j^{i^j} = C_0^{3^0}  \cdot  C_1^{3^1}  \cdot  C_2^{3^2}\end{math}
\end{center}

%--------------------------------------------------------------------------
\section{Tallying}
%--------------------------------------------------------------------------
Tallying is the process of counting the votes. Here the tallier uses their private keys to collectively compute the final tally, based on the valid ballots.\\


\noindent
\parahead{Homomorphic secret sharing} property ensures that each tallier will  be able to multiply the shares and decrypt. Note that \textit{i} is referring to tallier 1, tallier  2 and tallier 3 and \textit{j} is referring to voter 1, voter 2 and voter 3.:\\

\begin{infobox}[The tallier decrypts the shares and publishes a $DLEQ$ proof]
\begin{math}Y_i^*=(\prod\limits_{j=1}^{m} Y_{ij}) \ (mod\ 2 \cdot q+1) =y_i^{\sum\limits_{j=1}^m p_j(i)}\end{math}\\ \\
Following can be reduced by using the talliers private key. Note that we are raising to the multiplicative inverse of the private key in  \textit{q}:\\\\
\begin{math}y_i^{\sum\limits_{j=1}^m p_j(i)}=(G^{x_i})^{\sum\limits_{j=1}^m p_j(i)} = G^{x_i \sum\limits_{j=1}^m p_j(i)}= (G^{x_i \sum\limits_{j=1}^m p_j(i)})^{\frac{1}{x_i}}= G^{ \sum\limits_{j=1}^m p_j(i)}   \end{math}
\end{infobox}


\noindent
Note that besides decrypting the shares the tallier will publish a $DLEQ$ proof which shows that the decrypting was done correct. See figure \ref{fig:DLEQ_by_talliers} of the $DLEQ$ proof. Also note that we computing the inverse of the key $x_i$. To compute the inverse we will use Extended Euclidean algorithm.

\parahead{Example} on the multiplcation of the $Y_{ij}$
\begin{center}
Voter 1: \begin{math}Y_{1,1}=y_{1,1}^{p_1(1)} \end{math}, \begin{math}Y_{2,1}=y_{2,1}^{p_2(2)} \end{math},.., \begin{math}Y_{n,1}=y_n^{p_n(n)} \end{math}\\
Voter 2: \begin{math}Y_{1,2}=y_{1,2}^{p_1(1)} \end{math}, \begin{math}Y_{2,2}=y_{2,2}^{p_2(2)} \end{math},.., \begin{math}Y_{n,2}=y_n^{p_n(n)} \end{math}\\
Voter \textit{m}: \begin{math}Y_{1,m}=y_{1,m}^{p_1(1)} \end{math}, \begin{math}Y_{2,m}=y_{2,m}^{p_2(2)} \end{math},.., \begin{math}Y_{n,m}=y_n^{p_n(n)} \end{math}\\
\end{center}

\noindent
Example on $Y_i^*$ is each tallier can publish the following. Note that every talliers exponent is the evaluation in some point in a given polynomium \begin{math}h(1), h(2),..., h(n) \end{math}.
\begin{center}
Tallier 1 publish: \begin{math}Y_1^* = G^{ \sum\limits_{j=1}^m p_j(1)}   \end{math}\\
Tallier 2 publish: \begin{math}Y_2^* = G^{ \sum\limits_{j=1}^m p_j(2)}   \end{math}\\
Tallier \textit{n} publish: \begin{math}Y_n^* = G^{ \sum\limits_{j=1}^m p_j(n)}  \end{math}\\
\end{center}

\parahead{Lagrange interpolation} One can now use lambda, which is computed from the  Lagrange interpolation formular, \begin{math} \lambda_j \end{math} and multiply these values which will be the same as the sum of the exponents. See figure \ref{info:Computing_the_coefficients} \\

\noindent
The \begin{math}p_j(0) \end{math} correspond to the sum of the secret values of \textit{s} for the voters. This \begin{math}p_j(0)\end{math} corresponds to the evaluation of \begin{math}h(0)\end{math}. See figure  \ref{info:Applying_Lagrange_polynomial_interpolation}

\begin{infobox}[A master authority applies Lagrange interpolation]
\begin{math}(Y_1^*)^{\lambda_1  \cdot  \ (mod \ q)}  \cdot  (Y_2^*)^{\lambda_2 \ (mod \ q)}  \cdot  (Y_n^*)^{\lambda_n \ (mod \ q)} \ (mod \ 2 \cdot q+1) \end{math}\\


\noindent
We can substitute the $Y^*$ with $G$. The final result in the exponents is a evaluation in some polynomium in $0$ which corresponds to the sum of all random values $s$\\


\noindent
$=G^{ \sum\limits_{j=1}^m \lambda_j p_j(1)} \cdot G^{ \sum\limits_{j=1}^m \lambda_j p_j(2)} \cdot...\cdot G^{ \sum\limits_{j=1}^m \lambda_j p_j(n)} $\\
$=G^{ \sum\limits_{j=1}^m \lambda_j p_j(1) +  \sum\limits_{j=1}^m \lambda_j p_j(2) +...+  \sum\limits_{j=1}^m \lambda_j p_j(n)} $\\
$=G^{ \sum\limits_{j=1}^m (\lambda_j p_j(1)+\lambda_j p_j(2)+,...,+\lambda_{j}p_j(n))} = G^{ \sum\limits_{j=1}^m p_j(0)}= G^{ \sum\limits_{j=1}^m s_j}  $
\end{infobox}



\noindent
The last step is to isolate the votes and then compute the final result.


\noindent
\begin{infobox}[A master authority computes the votes]
By combining \begin{math}U_j \end{math} we obtain: \\ 
\begin{math} (\prod\limits_{j=1}^{m} U_{j}) \ (mod \ 2 \cdot q+1)=  G^{ \sum\limits_{j=1}^m s_j +v_j}\end{math} \\ \\
By dividing we isolate \textit{v}: \\ \\
\begin{math}\frac{G^{ \sum\limits_{j=1}^m s_j +v_j}}{{ G^{ \sum\limits_{j=1}^m s_j} }} =G^{ \sum\limits_{j=1}^m s_j +v_j -\sum\limits_{j=1}^m s_j} = G^{ \sum\limits_{j=1}^m v_j}  \end{math}
\end{infobox}


\parahead{Exhaustive search} To solve the computing of the votes one can compute \begin{math}G^0, G^1, G^3,..., G^{v_j} \end{math} by exhaustive search. 

\parahead{Baby-step giant-step} A more efficient algorithm is to use Baby-step giant-step algorithm. 


%--------------------------------------------------------------------------
\section{Proofs}
%--------------------------------------------------------------------------
In this section we will elaborate the mathematical justification of the interactive- and non interactive DLEQ and the $PROOF_U$. 

%----------------------------------------------------------------------
\subsection{DLEQ interactive proof between voters and verifier}
%----------------------------------------------------------------------
DLEQ stands for discrete logarithm equality and it proofs the exponent are equal \begin{math}X_i=g^{p(i)}  \land Y_i=y_i^{p(i)} \end{math} without revealing \begin{math}{p(i)} \end{math}. The verification of the shares in the interactive proof happens by the following  interaction between the prover and the verifier.

\parahead{The prover computes}  \begin{math}a_1=g^w  \land a_2=y_i^w,  w\in_R \Z_q \end{math} and sends $a_1$ and $a_2$ to the verifier. \textbf{ The verifer computes}  \begin{math}C\in_R \Z_q \end{math} and sends $C$ to the prover. \textbf{The prover computes}  \begin{math}r=w-p(i)  \cdot  C\end{math}. \textbf{The verifer computes}  if the following computation is true \begin{math}a_1 = g^r \cdot X_i^C \ \land \ a_2=y_i^r \cdot Y_i^C\end{math}.


\begin{figure}[H]
    \centering        
    
    $
    \begin{array}{l}
    \hline                      \
    \textbf{DLEQ protocol}      \\
    \hline                      \
    Public:  g,X_i,y_i,Y_i       \\
    \\
	\begin{array}{L{2cm}ccc}
        & \text{\textsf{Prover}} & & \text{\textsf{Verifier}} \\
        \hline
        Step \ 1 & w\in_R \Z_q & & \\
        & a_1=g^w     & & \\
        & a_2=y_i^w   & \xrightarrow{\hspace{1em}a_1 \land a_2\hspace{1em}} & \\
        Step \ 2 & & & C\in_R \Z_q \\
        & & \xleftarrow{\hspace{2em}C\hspace{2em}} & \\
        Step \ 3 & r=w-p(i)  \cdot  C    & & \\
        Step \ 4 & & \xrightarrow{\hspace{2em}r\hspace{2em}} & \begin{array}{c}
        checks \ if: \\      
        a_1 = g^r \cdot X_i^C \\ 
        a_2=y_i^r \cdot Y_i^C
        \end{array} \\
        \hline
    \end{array}
    \end{array}
    $    
    \caption{DLEQ interactive}
	\label{fig:DLEQ_interactive}
\end{figure}
	


\noindent
Note that the verifier sends a challenge \textit{C}, after the prover has computed \begin{math}a_1\end{math} and  \begin{math}a_2\end{math}. The check only passes if the prover used the same exponents. The proof shows that there exist some element \begin{math} \alpha\end{math} such that \begin{math}g^\alpha = X_i \ \land \ y_i^\alpha=Y_i \end{math} where  \begin{math} \alpha\end{math} is equal to  \begin{math} p(i)\end{math}. The following is an argument why DLEQ works.\\ 

\noindent
\parahead{Example} We will now show a concrete example why the challenge matters. Say that the challenge doesnt matter it will be the same as saying that the challenge is 1. 

\begin{infobox}[DLEQ without a challenge]
Prover \begin{math}\rightarrow\end{math} Verifier: \begin{math}a_1=g^6 \ \land\ a_2=y_i^7 \ \land \ X_i=g^2 \ \land \ Y_i=y_i^3 \end{math}\\
Verifier \begin{math}\rightarrow\end{math} Prover: \begin{math}C=1 \end{math}\\
Prover \begin{math}\rightarrow\end{math} Verifier: \begin{math}r=w-p(i)  \cdot  C = 6-2  \cdot  1= 4\end{math}\\
Verifier knows:  \begin{math}a_1=g^4  \cdot  X_i \ \land \ a_2=y_i^4  \cdot  Y_i \end{math}\\
Verifier checks if:  \begin{math}a_1 = g^4 \cdot X_i^C = g^4 \cdot g^{2 \cdot 1} = g^6\end{math}\\
Verifier checks if:  \begin{math} a_2=y_i^4  \cdot  Y_i^C = y_i^4  \cdot  y_i^{3 \cdot 1}= y_i^7 \end{math}
\end{infobox}

\noindent
One can see that both checks passes in this example despite that the prover is dishonest. Next example shows the verification with a challenge.

\begin{infobox}[DLEQ with a challenge]
Prover \begin{math}\rightarrow\end{math} Verifier: \begin{math}a_1=g^6 \ \land\ a_2=y_i^7 \ \land \ X_i=g^2 \ \land \ Y_i=y_i^3 \end{math}\\
Verifier \begin{math}\rightarrow\end{math} Prover: \begin{math}C=9\ (mod \ 5) \end{math}\\
Prover \begin{math}\rightarrow\end{math} Verifier: \begin{math}r=w-p(i)  \cdot  C = 6-2  \cdot  4 \ (mod \ 5)= 3\end{math}\\
Verifier knows:  \begin{math}a_1=g^3  \cdot  X_i \ \land \ a_2=y_i^3  \cdot  Y_i \end{math}\\
Verifier checks if:  \begin{math}a_1 = g^3 \cdot X_i^C = g^3 \cdot g^{2 \cdot 4} = g^3  \cdot  g^3 = g^6= g^1= g\end{math}\\
Verifier checks if:  \begin{math} a_2=y_i^3  \cdot  Y_i^C = y_i^3  \cdot  y_i^{3 \cdot 4}= y_i^3  \cdot  y_i^{1 \cdot 2}= y_i^3  \cdot  y_i^2= y^5=y^0= 1 \end{math}
\end{infobox}


\parahead{The mathematical justification} Verifier knows that the prover has raised \begin{math}g^w\end{math} and \begin{math}y_i^w\end{math}. We denote these exponents by \begin{math}w\end{math}    and \begin{math}w^{'}\end{math}. The verifier also knows that the prover has computed \begin{math}X_i=g^{p(i)}  \land Y_i=y_i^{p(i)} \end{math}. We denote the exponents by \begin{math}m_i\end{math} and \begin{math}m_i^{'}\end{math}. We know
\begin{math} a_1= g^w \end{math} equal to \begin{math}g^r  \cdot  X_i^C=g^r \cdot g^{m_i \cdot C} = g^{r+m_i \cdot C}\end{math} and
\begin{math} a_2= y_i^{w^{'}}\end{math} equal to \begin{math}y_i^r  \cdot  Y_i^C=y_i^r \cdot y_i^{m_i^{'} \cdot C} = y_i^{r+m_i^{'} \cdot C}\end{math}.
Based on this we can now write two inequalities. It is unlikely that if the prover was dishonest that these two checks passes. It may not be clear but in order for these inequalities to be true we can rewrite.

\begin{infobox}[DLEQ mathematical justification]
We know\\
\begin{math} a_1= g^w \end{math} equal to \begin{math}g^r  \cdot  X_i^C=g^r \cdot g^{m_i \cdot C} = g^{r+m_i \cdot C}\end{math}\\
\begin{math} a_2= y_i^{w^{'}}\end{math} equal to \begin{math}y_i^r  \cdot  Y_i^C=y_i^r \cdot y_i^{m_i^{'} \cdot C} = y_i^{r+m_i^{'} \cdot C}\end{math}\\\\
We can now write two inequalities\\
\begin{math} w= r+m_i  \cdot  C\ (mod\ q) \implies r= w-m_i \cdot C\ (mod\ q) \end{math}\\
\begin{math} w^{'}= r+m_i^{'}  \cdot  C\ (mod\ q) \implies r= w^{'}-m_i^{'}  \cdot  C\ (mod\ q) \end{math}\\\\
These two inequalities has to be equal, therefor we can rewrite\\
\begin{math}w-m_i \cdot C = w^i-m_i^{'}\ (mod\ q) \implies (w-w^i)-(m_i - m_i^{'})  \cdot  C = 0 \ (mod\ q) \end{math}
\end{infobox}


\noindent
The prover has to be honest if this equation must be true. It is very unlikely that, if the prover has been dishonest that he will succeed. Since the prover doesn't know the challenge the probability will be \begin{math} \frac{1}{q}\end{math}.


\begin{infobox}[Zero knowledge proof for the DLEQ]
\textcolor{red}{Question to Ignacio: we need to have clear separation on what is referring to what in our description of the proof}\\\\
\parahead{Completeness:} If the prover is honest it should pass or if the statement is true, the honest verifier (that is, one following the protocol properly)
will be convinced of this fact by an honest prover.\\\\
\parahead{Soundness:}If the statement is false then it should fail with large probability or if the statement is false, no cheating prover can convince
the honest verifier that it is true, except with some small probability?\\\\
\parahead{Zero-knowledge:} If the statement is true, no cheating verifier learns anything other than the fact that the statement is true.?
\end{infobox}

%----------------------------------------------------------------------
\subsection{DLEQ non-interactive proof between voters and verifier}
%----------------------------------------------------------------------
Here we present how one can turn an interactive proof to a non interactive proof.

\parahead{Fiat Shamir} This is also known as the Fiat Shamir where we are transforming a interactive proof into a non interactive proof (signature scheme). Instead of the verifier computes a challenge, the prover computes the challenge as a random function (hash).\\
\textcolor{red}{We will describe an non optimized and a optimized version - does this have a name??}\\


\parahead{The prover computes} \begin{math}a_1=g^w \ (mod\ q)  \land a_2=y_i^w \ (mod\ q),  w\in_R \Z_q \end{math}. Then the prover computes the hash \begin{math}C=H(X_i,Y_i,a_1,a_2) \end{math}. Then the prover computes  \begin{math}r=w-p(i)  \cdot  C \ (mod\ q)\end{math}. Last the prover publish \begin{math}a_1, a_2,r,C\end{math}. \\

\noindent
\parahead{The verifier computes} the following computations \begin{math}a_1 = g^r \cdot X_i^C  \ (mod\ q)\end{math} and \begin{math} a_2=y_i^r  \cdot  Y_i^C \ (mod\ q)\end{math} and \begin{math}C=H(X_i,Y_i,a_1,a_2)\end{math}.


\begin{figure}[H]
    \centering        
    
    $
    \begin{array}{l}
    \hline                      \
    \textbf{DLEQ protocol}      \\
    \hline                      \
    Public:  g,X_i,y_i,Y_i       \\
    \\
	\begin{array}{L{1.1cm}lcl}
        & \text{\textsf{Prover}} & & \text{\textsf{Verifier}} \\
        \hline
        Step \ 1    &           \begin{array}{l}
                                    w\in_R \Z_q             \\ 
                                    a_1=g^w \ (mod\ q)      \\ 
                                    a_2=y_i^w \ (mod\ q)    \\
                                    C=H(X_i,Y_i,a_1,a_2)    \\
                                    r=w-p(i)  \cdot  C \ (mod\ q)
                                \end{array}     &               & \\
                    &                   &               & \\
        Step \ 2    &                   &               \xrightarrow{\hspace{0.4em}a_1 , a_2 , r , C\hspace{0.4em}} & \begin{array}{l}
            checks \ if: \\      
            a_1 = g^r \cdot X_i^C \\ 
            a_2=y_i^r \cdot Y_i^C \\
            C=H(Xi,Yi,a1,a2)
        \end{array} \\
        \hline
    \end{array}
    \end{array}
    $    
    \caption{DLEQ non interactive}
	\label{fig:DLEQ_1}
\end{figure}


\noindent
 Note in above that there is no interaction between the prover and the verifier. In the following we will show how one can improve the amount of computation of the challenge \textit{C}. Instead of computing the challenge \textit{n} times, one can compute it once and reuse the challenge.





\begin{infobox}[DLEQ optimized]
DLEQ \begin{math}(g,X_i,y_i,Y_i) \end{math} for  \begin{math}1\leq i \leq n \end{math}  \\
Prover publish: \begin{math}a_{1,i}=g^{w_i} \ (mod\ q)  \land a_{2,i}=y_i^{w_i} \ (mod\ q),  w_i\in_R \Z_q \end{math}\\
Prover computes the hash: \begin{math}C=H(X_i,Y_i,...,X_n,Y_n,a_{1,1},a_{2,1},\\a_{1,2},a_{2,2},...,a_{1,n},a_{2,n})\end{math}\\
Prover computes \begin{math}r_i\end{math}:  \begin{math}r_i=w_i-p(i)  \cdot  C \ (mod\ q)\end{math}\\
Prover publish:  \begin{math}r_i,C\end{math}\\
Verifier checks if:  \begin{math}a_{1,i} = g^{r,i} \cdot X_i^C \ (mod\ q) \end{math}\\
Verifier checks if:  \begin{math} a_{2,i}=y_i^{r_{i}}  \cdot  Y_i^C \ (mod\ q)\end{math}\\ 
Verifier checks if:  \begin{math}C=H(X_i,Y_i,...,X_n,Y_n,a_{1,1},a_{2,1},a_{1,2},a_{2,2},...,a_{1,n},a_{2,n})\end{math}

\label{info:DLEQ_optimized}
\end{infobox}

\parahead{Example} Hence the hash contains all the  \begin{math}a_i \end{math} the prover will compute this proof once for all  \begin{math}p(i) \end{math} which improve efficiency. This means that if there are 3 voters, then the above computation has to be done 3 times for every tallier, but same hash can be computed once for every tallier. 

\begin{infobox}[DLEQ computation for 3 voters]
Prover publish:  $a_{1,1},a_{2,1},a_{1,2},a_{2,2},a_{1,3},a_{2,3}$ \\
Prover publish:                   $C,r_1,r_2,r_3$ \\
Prover selects:  $w_1, w_2,w_3\in_R \Z_q$ \\
Prover computes: $a_{1,i}=g^w_i \ (mod\ q) \land a_{2,i}=y_i^{w_i} \ (mod\ q)$ \\
Prover computes the hash:  $C=H(X_i,Y_i,...,X_n,Y_n,a_{1,1},a_{2,1},$\\
$a_{1,2},a_{2,2},...,a_{1,n},a_{2,n})$\\
Prover computes  $\ r_i:  r_i=w_i-p(i)  \cdot  C \ (mod\ q)$ \\\\
The verification contains the following computation: \\
Verifier checks if: $a_{1,i} = g^{r,i} \cdot X_i^C \ (mod\ q) $        \\
Verifier checks if: $a_{2,i} =y_i^{r_{i}}  \cdot  Y_i^C \ (mod\ q)  $   \\
Verifier checks if: $C=H(X_i,Y_i,...,X_n,Y_n,a_{1,1},a_{2,1},a_{1,2},a_{2,2},...,a_{1,n},a_{2,n})$

\end{infobox}



\begin{infobox}[Zero knowledge proof for the DLEQ]
\textcolor{red}{Question to Ignacio: we need to have clear separation on what is referring to what in our description of the proof}\\\\
\parahead{Completeness:} If the prover is honest it should pass or if the statement is true, the honest verifier (that is, one following the protocol properly)
will be convinced of this fact by an honest prover.\\\\
\parahead{Soundness:}If the statement is false then it should fail with large probability or if the statement is false, no cheating prover can convince
the honest verifier that it is true, except with some small probability?\\\\
\parahead{Zero-knowledge:} If the statement is true, no cheating verifier learns anything other than the fact that the statement is true.?
\end{infobox}

%----------------------------------------------------------------------
\subsection{Description of $ \mathbf{PROOF_U} $}
%----------------------------------------------------------------------
In this section we will show the interactive proof. Through Fiat–Shamir  we transform an interactive proof of knowledge into a non-interactive proof of knowledge. First of all the voter proves that there is consistency between the exponents of how \begin{math}U\end{math} and \begin{math}C_0\end{math} is constructed from \begin{math}U=G^{s+v}\end{math} and \begin{math}C_0 = g^s\end{math}. The exponents only differs from the vote hence 0 or 1.\\

\begin{figure}[H]
    \centering        
    
    $
    \begin{array}{l}
    \hline                      \
    \textbf{$PROOF_U$ protocol}      \\
    \hline                      \
    Public:  U=G^{s+v},\ C_0=g^s       \\
    \\
	\begin{array}{L{1.4cm}lcr}
        & \text{\textsf{Prover}} & \text{\textsf{Verifier}} \\
        \hline
        Step \ 1a   &           \begin{array}{l}
                                    v=0             \\ 
                                    w\in_R \{1,...,q-1\}, \\ 
                                    r_1\in_R\{1,...,q-1\},\\
                                    d_1\in_R\{1,...,q-1\},      \\ 
                                    a_0 = g^w,\\ 
                                    a_1 = g^{r_1} \cdot C^{d_1}_0,\\ 
                                    b_0 = G^w,  \\
                                    b_1 = G^{r_1}  \cdot  (\frac{U}{G^{1-v}})^{d_1} = G^{r_1}  \cdot  (\frac{U}{G})^{d_1} \\
                                \end{array}     &               & \\
                                \\
                    &                   \xrightarrow{\hspace{1em}a_0, a_1, b_0, b_1\hspace{1em}} \ \textbf{Publish to bulletin}&  & \\                                
                    &                   &               & \\
        Step \ 1b   &           \begin{array}{l}
                                    v=1             \\ 
                                    w\in_R \{1,...,q-1\},\\ r_0\in_R\{1,...,q-1\},\\
                                    d_0\in_R\{1,...,q-1\},      \\ 
                                    a_0 = g^{r_0} \cdot C^{d_0}_0,\\
                                    a_1 = g^w,\\
                                    b_0 = G^{r_0}  \cdot  (\frac{U}{G^{1-v}})^{d_0}= G^{r_0}  \cdot  U^{d_0} ,\\
                                    b_1 = G^w\\
                                \end{array}     &               & \\
                    &                   &               & \\
                    \\
                    &                   \xrightarrow{\hspace{1em}a_0, a_1, b_0, b_1\hspace{1em}} \ \textbf{Publish to bulletin} & \\
        Step \ 2    &                    & \begin{array}{l}
                                \textbf{Create \ challenge:} \\      
                                C\in_R \{0,...,q-1\} \\ 
                                \xleftarrow{\hspace{2em}C\hspace{2em}}\\
                                \end{array}  \\
        Step \ 3a   &          \begin{array}{l}
                                   v=0             \\ 
                                   d_0= C-d_1\ mod\ q,\\
                                   r_0=w-s \cdot d_0 \ mod\ q\\  
                                \end{array}     &               & \\
                                \\
                    &                   \xrightarrow{\hspace{1em}d_0,\ r_0,\ d_1,\ r_1\hspace{1em}} \ \textbf{Publish to bulletin}&  & \\                                
                    &                   &               & \\
        Step \ 3b   &           \begin{array}{l}
                                    v=1             \\ 
                                    d_1= C-d_0\ mod\ q, \\
                                    r_1=w-s \cdot d_1 \ mod\ q\\ 
                                \end{array}     &               & \\
                                \\
                    &                   \xrightarrow{\hspace{1em}d_0,\ r_0,\ d_1,\ r_1\hspace{1em}} \ \textbf{Publish to bulletin}&  &\\
                    &                   &               & \\
        Step \ 4   &                    & \begin{array}{l}
                                \textbf{Verification:} \\      
                                C = d_1 + d_0,\\
                                a_0=g^{r_0}  \cdot  C^{d_1}_0,\\
                                b_0 = G^{r_0} \cdot U^{d_0},\\
                                a_1=g^{r_1}  \cdot  C^{d_1}_0,\\
                                b_1= G^{r_1}  \cdot (\frac{U}{G})^{d_1} \\ 
        \end{array} \\
        \hline
    \end{array}
    \end{array}
    $    
    \caption{$PROOF_U$}
	\label{fig:PROOF_U}
\end{figure}

\parahead{Explanation of the protocol} In step 1 the voter publish \begin{math}a_0,\ b_0,\ a_1,\ b_1\end{math}. Note that the difference between voting 0 or 1 is just by swapping the values between the variables \begin{math}a_0,\ b_0\end{math} and \begin{math}a_1,\ b_1\end{math}. The point is that the verifier will not be able to distinguish the value of the vote and thereby gain knowledge about the vote.

\noindent
\begin{infobox}[Step 1]
Voter votes 0 and creates and publish:\\
\begin{math}v=0,\ w\in_R \{1,...,q-1\},\ r_1\in_R\{1,...,q-1\},\ d_1\in_R\{1,...,q-1\}\end{math}\\
Publish: \begin{math}a_0 = g^w,\ b_0 = G^w,\ a_1 = g^{r_1} \cdot C^{d_1}_0,\ b_1 = G^{r_1}  \cdot  (\frac{U}{G^{1-v}})^{d_1} = G^{r_1}  \cdot  (\frac{U}{G})^{d_1} \end{math}\\\\
Voter votes 1 and creates and publish\\
\begin{math}v=1,\ w\in_R \{1,...,q-1\},\ r_0\in_R\{1,...,q-1\},\ d_0\in_R\{1,...,q-1\}\end{math}\\
Publish: \begin{math}a_1 = g^w,\ b_1 = G^w,\ a_0 = g^{r_0} \cdot C^{d_0}_0,\ b_0 = G^{r_0}  \cdot  (\frac{U}{G^{1-v}})^{d_0}=  G^{r_0}  \cdot  U^{d_0} \end{math}
\end{infobox}

\noindent
\begin{infobox}[Step 2]
The verifier creates a challenge \begin{math}C\in_R \{0,...,q-1\}\end{math} to the voter.
\end{infobox}

\noindent
The outcome from step 3 is that the voter publish \begin{math}d_0,\ r_0,\ d_1,\ r_1\end{math}. Note that the voters computation depends on the challenge from the interaction between the verifier.

\noindent
\begin{infobox}[Step 3]
Voter votes 0 computes\\
\begin{math}v=0,\ d_0= C-d_1\ mod\ q, \ r_0=w-s \cdot d_0 \ mod\ q\end{math}\\\\
Voter votes 1 computes\\
\begin{math}v=1,\ d_1= C-d_0\ mod\ q, \ r_1=w-s \cdot d_1 \ mod\ q\end{math}
\end{infobox}

\noindent
In step 4 the verifier will be able computes and verify consistency.

\noindent
\begin{infobox}[Step 4]
\begin{math}C = d_1 + d_0,\ a_0=g^{r_0}  \cdot  C^{d_1}_0,\ b_0 = G^{r_0} \cdot U^{d_0},\ a_1=g^{r_1}  \cdot  C^{d_1}_0,\ b_1= G^{r_1}  \cdot (\frac{U}{G})^{d_1}\end{math}
\end{infobox}

\parahead{Non-interactive proof} We can turn this into a non-interactive proof by replacing step 2 with the voter using a hash function and thereby avoiding interaction with the verifier. The prover will then compute a hash of \begin{math}C=H(U,\ C_0,\ a_0,\ b_0,\ a_1,\ b_1) \end{math}.\\

\parahead{Mathematical justification} In the following we derive the mathematical justification from step 4. Here we will replace with earlier expression from above and replace by the actual value of the vote. What this means is that the math depends on the value of the vote.\\



\begin{infobox}[Explanation of \begin{math}a_0=g^{r_0}  \cdot  C^{d_1}_0\end{math}]
vote=0, show that \begin{math}a_0=g^{r_0}  \cdot  C^{d_0}_0 = g^w \end{math} is well constructed
\begin{align*}
    a_0 &=g^{r_0}  \cdot  C^{d_0}_0     \\ 
        &= g^{w-sd_0} \cdot  g^{sd_0}   \\
        &= g^{w-sd_0+ sd_0}             \\
        &= g^w                          
\end{align*}
vote =1, this is trivial because \begin{math}a_0 \end{math}  is constructed from \begin{math}a_0=g^{r_0}  \cdot  C^{d_0}_0 \end{math}
\end{infobox}


\begin{infobox}[Explanation of \begin{math}b_0 = G^{r_0} \cdot U^{d_0}\end{math}]
vote=0, show that \begin{math}b_0 = G^{r_0} \cdot U^{d_0} = G^w \end{math} is well constructed
\begin{align*}
    b_0 &= G^{r_0} \cdot U^{d_0}                    \\
        &= G^{w-sd_0} \cdot G^{(s+0) \cdot d_{0}}   \\
        &= G^{w-sd_0} \cdot G^{sd_{0}}              \\
        &= G^w                                          
\end{align*}
vote 1, show that \begin{math}b_0 = G^{r_0} \cdot (\frac{U}{G^{1-1}})^{d_0}=  G^{r_0} \cdot U^{d_0} \end{math}
\begin{align*}
    b_0 &= G^{r_0} \cdot (\frac{U}{G^{1-v}})^{d_0}  \\
        &= G^{r_0} \cdot (\frac{U}{G^{0}})^{d_0}    \\
        &= G^{r_0} \cdot (\frac{U}{1})^{d_0}        \\
        &= G^{r_0} \cdot U^{d_0}  
\end{align*}
\end{infobox}



\begin{infobox}[Explanation of \begin{math}a_1=g^{r_1}  \cdot  C^{d_1}_0\end{math}]
vote= 1, show that \begin{math}a_1=g^{r_1}  \cdot  C^{d_1}_0 = g^w \end{math} is well constructed
\begin{align*}
    a_1 &= g^{r_1}  \cdot  C^{d_1}_0    \\
        &= g^{w-sd_1} \cdot  g^{sd_1}   \\
        &= g^{w-sd_1+ sd_1}             \\
        &= g^w
\end{align*}
vote =0, this is trivial because \begin{math}a_1 \end{math} is constructed from \begin{math}a_1=g^{r_1}  \cdot  C^{d_1}_0 \end{math}
\end{infobox}


\begin{infobox}[Explanation of \begin{math}b_1= G^{r_1}  \cdot (\frac{U}{G})^{d_1}\end{math}]
vote=1, show that \begin{math}b_1= G^{r_1}  \cdot (\frac{U}{G})^{d_1} = G^W\end{math} is well constructed \begin{align*}
    b_1 &= G^{r_1}  \cdot (\frac{U}{G})^{d_1}                   \\
        &= G^{w-sd_1} \cdot (U \cdot G^{-1})^{d_1}              \\
        &= G^{w-sd_1} \cdot  (G^{s+1})^{d_0} \cdot G^{-d_1}     \\
        &= G^{w-sd_1} \cdot  G^{sd_0+d_0} \cdot G^{-d_1}        \\
        &= G^W
\end{align*}
vote=0, show that  \begin{math}b_1= G^{r_1}  \cdot (\frac{U}{G^{1-v}})^{d_1} = G^{r_1}  \cdot (\frac{U}{G})^{d_1}\end{math} is well constructed
\begin{align*}
    b_1 &= G^{r_1}  \cdot (\frac{U}{G^{1-v}})^{d_1}             \\ 
        &= G^{r_1}  \cdot (\frac{U}{G^{1-0}})^{d_1}             \\
        &= G^{r_1}  \cdot (\frac{U}{G})^{d_1}
\end{align*}
\end{infobox}



\begin{infobox}[Zero knowledge proof for the $PROOF_U$]
\textcolor{red}{Question to Ignacio: we need to have clear separation on what is referring to what in our description of the proof}\\\\
\parahead{Completeness:} If the prover is honest it should pass or if the statement is true, the honest verifier (that is, one following the protocol properly)
will be convinced of this fact by an honest prover.\\\\
\parahead{Soundness:}If the statement is false then it should fail with large probability or if the statement is false, no cheating prover can convince
the honest verifier that it is true, except with some small probability?\\\\
\parahead{Zero-knowledge:} If the statement is true, no cheating verifier learns anything other than the fact that the statement is true.?
\end{infobox}

%----------------------------------------------------------------------
\subsection{DLEQ proof by the talliers}
%----------------------------------------------------------------------
The tallier will do computations on each of their shares. Here each tallier uses the DLEQ to prove that the decryption of their shares is done correctly. We will first show the interactive proof and then transform it to a non-interactive proof. The input values are \begin{math}(G,\ y_i,\ S_i, Y_i)\end{math} where \begin{math}G = y_i^{x_i}\end{math} and  \begin{math} Y_i=S_i^{x_i}\end{math}. We have some initial values
 \begin{math}g_1 =G,\ h_i=y_i,\ g_2=S_i,\ h_2=Y_i,\ \alpha=x_i \end{math} and \begin{math}w \in_R \{0,...,q-1\}\end{math}.\\
 
 
 
\noindent 
In step 1 the prover computes \begin{math}a_1,\ a_2=S_i^w\end{math}. In step 2
the verifier creates a challenge \begin{math}C\end{math}. In step 3 the tallier computes \begin{math}r=w-C \cdot x_i\end{math}. In step 4 the verifier computes \begin{math}a_1=G^r \cdot y_i^C,\ a_2=S_i^r \cdot Y_i^C\end{math}.
 
 
 
\begin{figure}[H]
    \centering        
    
    $
    \begin{array}{l}
    \hline                      \
    \textbf{DLEQ protocol by the talliers}      \\
    \hline                      \
    Input:  G,\ y_i,\ S_i, \ Y_i \ where \ G = y_i^{x_i} \land Y_i=S_i^{x_i}     \\
    Output: 0 \ or \ 1
    \\
	\begin{array}{L{2cm}ccc}
        & \text{\textsf{Prover}} & & \text{\textsf{Verifier}} \\
        \hline
        Step \ 1 & w\in_R \Z_q & & \\
        & a_1=g^w     & & \\
        & a_2=y_i^w   & \xrightarrow{\hspace{1em}a_1 \land a_2\hspace{1em}} & \\
        Step \ 2 & & & C\in_R \Z_q \\
        & & \xleftarrow{\hspace{2em}C\hspace{2em}} & \\
        Step \ 3 & r=w-p(i)  \cdot  C    & & \\
        Step \ 4 & & \xrightarrow{\hspace{2em}r\hspace{2em}} & \begin{array}{c}
        checks \ if: \\      
        a_1 = g^r \cdot X_i^C \\ 
        a_2=y_i^r \cdot Y_i^C
        \end{array} \\
        \hline
    \end{array}
    \end{array}
    $    
    \caption{DLEQ}
	\label{fig:DLEQ_by_talliers}
\end{figure} 
 
\parahead{Non-interactive proof} Note the interaction in step 2 where the verifier create a challenge to the server. Through Fiat–Shamir we transform an interactive proof of knowledge into a non-interactive proof of knowledge. We replace step 2 with a hashing algorithm  \begin{math}C=H(G,\ y_i,\ S_i,\ Y_i,\ a_1,\ a_2)\end{math}.\\
 
\parahead{Mathematical justification} For the mathematical justification in step 4, we can do the following computation on \begin{math}a_1=G^w\end{math} and \begin{math}a_2=S_i^w\end{math}.\\\\
Let \begin{math}a_1=G^w\end{math} we show that \begin{math}G^r \cdot y_i^C=G^r \cdot y_{x_i}^C=G^{r+x_iC}=G^{w-Cx_i+x_iC}=G^w\end{math}\\\\
Let \begin{math}a_2=S_i^w\end{math} we show that
\begin{math}S_i^r \cdot Y_i^C=S^r \cdot Y_{x_i^C}=S^{w-Cx_i} \cdot S_i^{x_i \cdot C}=S^{w-Cx_i+x_i \cdot C}=S_i^w\end{math}



\begin{infobox}[Zero knowledge proof for the $DLEQ$ for the tallier]
\textcolor{red}{Question to Ignacio: we need to have clear separation on what is referring to what in our description of the proof}\\\\
\parahead{Completeness:} If the prover is honest it should pass or if the statement is true, the honest verifier (that is, one following the protocol properly)
will be convinced of this fact by an honest prover.\\\\
\parahead{Soundness:}If the statement is false then it should fail with large probability or if the statement is false, no cheating prover can convince
the honest verifier that it is true, except with some small probability?\\\\
\parahead{Zero-knowledge:} If the statement is true, no cheating verifier learns anything other than the fact that the statement is true.?
\end{infobox}
%************************* Proof end