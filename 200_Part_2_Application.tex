\noindent
This chapter will describe and discuss our theoretical aspects behind our implementation of the electronic voting application.  


















\noindent
Quatilty attribute\\
Quatilty attribute drivers\\
QAW og Quality attribute scenario : skalerbarhed og sikkerhed\\
Tactics\\
3+1 viewpoints\\
Distributed patterns\\
Microservices: distributed randomness\\
Sikkerhed på en browser\\
Toplogi: stjerner ingen snakker med hinanden\\
Beslutning omkring valg af kodeplatform... er det arktikturmæssig beslutning... Måske decisional krutchen ??\\\\

\noindent
klient (voter)\\
tallier\\
oberserver (master authority)\\
bullutin board (server)\\
\noindent
Initielt\\
En klient melder sig til bullutin board. Får tildelt public værdier. \\
En tallier melder sig til bullutin board. \\\\
\noindent
Ballotcasting -> Nu starter voting ->  $DLEQ$ og $PROOF_U$ -> indenfor deadline\\
Tallying -> Optællingsfasen for hver \\
Master authority -> Endelig summering -> brute force -> publish samlet stemme -> en vote som ikke er valideret tæller ikke med -> logning af beregning \\\\

\noindent
Issue tables\\
Diskussion/ta stilling \\
Remote randomness / webbrowser / .NET\\
https\\
man må ikke dobbelt vote\\
store tal\\
hardware krav