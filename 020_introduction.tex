\section{Introduction}
In this master thesis we will study how to implement an  electronic voting scheme application which is based on some tools from cryptography where the most important tool is the Publicly Verifiable Secret Sharing (PVSS) protocol. The main paper for this protocol will be Berry Schoenmakers paper "A Simple Publicly Verifiable Secret Sharing Scheme and its Application to Electronic Voting" \cite{Schoenmakers1999}. There are theoretical papers about how participants can communicate secure. For different reasons there are fewer papers which propose how to implement their theoretical results. 

\parahead{Secret sharing} is a tool which is the PVSS protocol uses. Secret sharing is a way of distributing a secret among several servers. Secret sharing is where a dealer has a secret. The dealer shares pieces of the secret to each of the players. If there are a large number of players they will be able to reconstruct the secret. More technically the idea is to hide a secret inside of a polynomial so that given certain partial information of the polynomial we can recover the secret that was hidden in it.

\parahead{A Multiparty computation protocol (MPC)} is a protocol which uses secret sharing and allows several parties to compute some function on some private inputs, in such a way that they learn the result but not the inputs from the other players. There are two types of MPC protocols. First there are MPC protocols which are secure against passive corruption which assumes that everyone is honest and sends correct data. This is whats called secure against passive corruption. Then there are MPC protocols which are secure against active corruption where adversary is able to send incorrect data. Here we need more advanced protocols like the Verifiable secret sharing protocol (VSS) and Zero knowledge proofs. Here the protocol allows the involved participants to verify their shares as consistent. This means that the VSS will be able to detect incorrect shares.

\parahead{In the Public verifiable secret sharing} is an extension to the VSS protocol where the goal is not just that the participants can verify their own shares, but that anybody can verify the correctness of the transmitted data. To get a understanding of the protocol we will describe the formal parts behind the protocol. 

\parahead{Practical implementation} We will design and implement distributed electronic voting solution which implements the PVSS protocol. In this solution we will have focus on designing a secure and scalable distributed architecture. Using known software design principle from Bass, Bærbak and Nygaard, we discuss and reflect on different solutions.

