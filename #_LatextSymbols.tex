\documentclass{article}
\usepackage{amssymb}
\usepackage{amsmath}
\usepackage{amsthm} 
\usepackage{url}
\usepackage{xspace}
\usepackage{graphicx}
\usepackage{float}
\usepackage[danish]{babel}
\usepackage[utf8]{inputenc}
\usepackage[T1]{fontenc}
\usepackage[vlined]{algorithm2e}
\usepackage{pgfplots}

\newcommand{\svec}{\ensuremath{\mathbf{s}}\xspace}


\begin{document}
\begin{math}
\frac{abc}{xyz}
 \end{math}    
 

Figur \ref{fig:DLEQ_1}

% Information vedr.  pgfplots // grafer: 
% https://www.maths.adelaide.edu.au/anthony.roberts/LaTeX/pgfplotBasics.pdf

\begin{tikzpicture}
\begin{axis}[
        xmin=0, xmax=4,
        ymin=0, ymax=4,
        axis lines=center,
        axis on top=true,
        domain=0:1,
        title style={at={(0.5,-0.15)},anchor=north},
        title={\textbf{Kaspers graf}},
    ]
    \addplot[
        scatter,
        only marks,
        point meta=explicit symbolic,
        scatter/classes={a={black}},
    ]
    table[meta=label] {
        x       y       label
        1       2       a
        2       4       a
        3       3       a
        4       4       a
    };
\end{axis}
\end{tikzpicture} 
 
 
\begin{figure}[H]
	\begin{center}
 	\begin{tabular}{ l c r }
		\bf{HB+ Protokol} \\
		\\
		Offentlig parameter: $k, \tau, q$ \\
		Hemmelig nøgle: $\svec_1, \svec_2 \in \{0,1\}^n$ \\
		Parameter: $L = 0$ \\
		Følgende procedure gentages $q$ gange \\
		\hline
		\bf{Læser} & & \multicolumn{1}{l}{\bf{Tag}} \\
		& & \multicolumn{1}{l}{Vælger udfordringen $\bvec \in_R \{0,1\}^n$} \\
		& $\overset{\bvec}{\leftarrow}$ & \\
		Vælger udfordringen $\avec \in_R \{0,1\}^k$ & & \\
		& $\overset{\avec}{\rightarrow}$ & \\
		& & \multicolumn{1}{l}{$e \in_{\ber} \{0,1\}$} \\
		& & \multicolumn{1}{l}{Beregner svaret $r \leftarrow \dotp{\avec}{\svec_1} \oplus \dotp{\bvec}{\svec_2} \oplus e$ } \\
		& $\overset{r}{\leftarrow}$ &  \\
		Hvis $r = \dotp{\avec}{\svec_1} \oplus \dotp{\bvec}{\svec_2}$ så tælle vi $L = L + 1$ & & \\
		\hline
		Tag'et blive accepteret hvis $L > (1-\tau)q$ & &
	\end{tabular}
	\end{center}
	\caption{HB+ protokol}
	\label{fig:HB+}
\end{figure} 

\clearpage
\begin{center}
\begin{algorithm}[H]
\KwIn{a set $V$ of $q$ queries $(v_i, c_i) \in \{0,1\}^{k+1}$ from the LPN Oracle, values $a, b$ such that $k = ab$}
\KwOut{values $\svec_1,...\svec_b$}
\SetKw{KwAnd}{and}
\SetKw{KwOr}{or}
\Begin{
Partitions the positions $a$ into disjoint $q_1 \cup...\cup q_{a-1}$ with $q_i$ of size $b$\;
\For{$i=1$ to $a-1$}{
  Partition $V = V_1 \cup ... \cup V_{2^b}$ s.t.vectors in $V_j$ have the same bit values on $q_i$\;
  \For{\textbf{each} $V_j$}{
    Choose a random $(v^*,c^*) \in V_j$ as a representative vector\;
    Replace each $(v,c)$ by $(v,c) \oplus (v^*,c^*), (v,c) \in V_j$ for $(v,c) \neq (v^*,c^*)$\;
    Discard $(v^*,c^*)$ from $V_j$\;
  }
  $V = V_1 \cup ... \cup V_{2^b}$\; 
  }
$f(x)=\sum_{i} 1_{v'_i}=x(-1)^{c'_i}$\;
$f(v)=\sum_{x} (-1)^{\dotp{v}{x}}f(x)$\;
$(\svec_1,...,\svec_b) = arg max(f(v))$
\Return{$\svec_1,...,\svec_b$}
}
\caption{LF1 Algoritmen som beskreven i [3]}
\end{algorithm}
\end{center}


\begin{thebibliography}{9}

\bibitem{levfou05}
  Eric Levieil and Pierre-Alain Fouque,
  \emph{An improved LPN algorithm},
  Paris Cedex 05, France, 2005.
  
\bibitem{BKW03}
  A. Blum, A. Kalai and H. Wasserman.
  \emph{Noise-tolerant Learning, the Parity Problem, and the statistical Query Problem},
  Journal of the ACM 50,4, July 2003, pp. 506-519.
  
\bibitem{BOTRVA}
  Sonia Bogos, Florian Tramér, and Serge Vaudenay
  \emph{On Solving LPN using BKW and Variants},
  {IACR} Cryptology ePrint Archive, 2015.  
  

\end{thebibliography}
 
\end{document}





    


