\section{Modular arithmetic} \label{sec:modular_arithmetic} 
\say{Modular arithmetic is a system of arithmetic for integers, where numbers \say{wrap around} upon reaching a certain value—the modulus} - Wikipedia \\

\noindent
An intuitive example of modular arithmetic usages is given a 12-hour clock. let the clock be 10:00 now, then in 5 hours the clock will show 3:00 and not 15:00. 3 is the remainder of 15 modulus 12. We can define this as.  \\

%**************************************Definition Modulo Operation Start
\begin{defi}[\textbf{Modulo Operation}]
Let \begin{math} a, r,m \in  \mathbb{Z}\end{math} (where \begin{math} \mathbb{Z}\end{math} is a set of all integers) and \begin{math} m > 0\end{math} and we write  
\begin{center} \begin{math} a = r \ mod \ m\end{math} \end{center}
if \begin{math}m \end{math} divides \begin{math} a - r \end{math}.\\
\begin{math}m \end{math} is called the modulus and \begin{math}r \end{math} is called the remainder.
\end{defi}
%**************************************Definition Modulo Operation End

\parahead{Computing the remainder} By example we can compute the remainder according to the definition. Given: \begin{math} a, m \in \mathbb{Z} \end{math} we compute the remainder by the following fomular:  \begin{math} a = qm +r \end{math}. This example illustrate that the remainder is not unique. Though the remainder will be unique in set $[0,$ m $)$.

\noindent
\begin{alignat*}{8}
42 &= 4 \cdot 9 +6 &&\implies r &&= 6 ,\ &&\text{by \ definition}: \  &&(42-6) &&= 36 ,  9| 36 \\
42 &= 3 \cdot 9 +15 &&\implies r &&= 15 ,\ &&\text{by \ definition}: \ &&(42-15) &&= 27 ,  9| 27 \\
42 &= 5 \cdot 9 +(-3) &&\implies r &&= -3 ,\ &&\text{by \ definition}: \ &&(42-(-3)) &&= 45 ,  9| 45 
\end{alignat*}




\parahead{Equivalence classes} Above can also be written with the modulo operator. Here we show that all have different remainder but are in the same equivalence class modulo \textit{9}. This means that all members of a given equivalence class behave equivalently. Note also that one can compute with negative integers.

\begin{align*}
42 &= 6 \ mod \ 9 \\
42 &= 15 \ mod \ 9 \\
42 &= -3 \ mod \ 9 
\end{align*}

\parahead{Computing the inverse}
As we will see later, computing the inverse becomes an important part in this protocol for how to divide modulo an integer arithmetically. For this we have the Extended Euclidean algorithm which allows us to compute the inverse modular some integers. This computation results in computing the inverse.


\parahead{Extended Euclidean algorithm}  To compute the inverse means that we want to compute the following \begin{math}a\ \cdot \ x \ mod \ q = 1 \end{math} where $a, \ x, \ q$ are integers and $x$ is the inverse of $a$. The condition for the existence of the inverse is that the $gcd(q,\ a)= 1$. This means that the only number which divides both $q$ and $a$ is $1$. So if $gcd(q,\ a)= 1$ the Extended Euclidean Algorithm computes the inverse of $a$.\\


\parahead{Explanation of the Extended Euclidean algorithm (EEA)} Since we will use an implementation in our electronic voting application we will explain the EEA by example. In order to explain the EEA we will explain the regular Euclidean algorithm (EA). The first two lines (6-7) in the algorithm is the EA. It turns out that if the $gcd(n,a)=1$  and we give the EEA $gcd(n,a)$, where $n$ is the modulo integer and $a$ is an integer, then the $t$ parameter will be the inverse of $a$. 
\cite{Paar}

%**************************************Pseudocode Euclidean algorithm start
\begin{center}
\begin{algorithm}[H]
\caption{Extended Euclidean Algorithm (EEA)\label{alg}}

\SetKwRepeat{Do}{do}{while}

\KwIn{positive integers $r_0 $ and $r_1 $ with $r_0 $ > $r_1 $}
\KwOut{gcd($r_0 $, $r_1 $), as well as s and t such that gcd($r_0$, $r_1$) = $s \cdot r_0+t \cdot r_1 $.}

\textbf{Initialization:} \\
$
\begin{array}{ll}
    s_0 = 1   & t_0 = 0 \\
    s_1 = 0   & t_1 = 1 \\
      i = 1     &           \\
\end{array}                 
$

\Begin{
    \Do{$r_i \neq 0 $}{
    $i = i+1 $\\
    $r_i = r_{i-2} \ mod \ r_{i-1} $\\
    \begin{math} \mathbf{ q_{i-1} = ( r_{i-2}-r_{i} ) / r_{i-1} }  \end{math}\\
    $s_i = s_{i-2}-q_{i-1} \cdot s_{i-1} $\\
    $t_i = t_{i-2}-q_{i-1} \cdot t_{i-1} $
    }

\Return{ \\
    gcd($r_0 $, $r_1 $) = $r_{i-1} $\\
    s = $s_{i-1} $\\
    t = $t_{i-1} $
    }
}
\end{algorithm}
\end{center}


%**************************************Pseudocode Euclidean algorithm end
\noindent
The EA works that given to integers $r_0 = 911$ and $r_1 = 301$. The gcd is computed by reducing the problem of finding the gcd of two given numbers to that of the gcd of two smaller numbers 


\begin{center}
\begin{tabular}{|ll|lll| } 
\hline
$911$ & $= 3 \cdot 301 +8$& $gcd(911,301)$ & $= gcd(301,8)$ & \\ 
\hline
$301$ & $= 37 \cdot 8+5$ & $gcd(301,8)$ & $= gcd(8,5)$ &\\ 
\hline
$8$ & $= 1 \cdot 5+3$ & $gcd(8,5)$ & $= gcd(5,3)$ & \\ 
\hline
$5$ & $= 1 \cdot 3+2$ & $gcd(5,3)$ & $= gcd(3,2)$ & \\ 
\hline
$3$ & $= 1 \cdot 2+1$ & $gcd(3,2)$ & $= gcd(2,1)$ & \\ 
\hline
$2$ & $= 1 \cdot 1+1$ & $gcd(2,1)$ & $= gcd(1,1)$ & \\ 
\hline
$1$ & $= 1 \cdot 1+0$ & $gcd(1,1)$ & $= gcd(1,0)$ & $= 1$ \\ 
\hline
\end{tabular}
\end{center}


\noindent
\parahead{Example Extended Euclidean algorithm} We will now extend the EA example with EEA with the same values $r_0 = 911$ and $r_1 = 301$ . Here the main point is that the last iteration we compute the parameter $t$, from two previous iterations. This $t$ parameter is the inverse of $r_1$. On the left-hand side, we compute the standard Euclidean algorithm, i.e., we compute new remainders $r_2$, $r_3$, ... Also, we have to compute the integer quotient $q_{i-1}$ in every iteration. On the right-hand side we compute the coefficients $s_i$ and $t_i$ such that $r_i = s_i r_0 + t_i r_1$. 

\begin{center}
\begin{tabular}{|l|l|p{5cm}| } 
\hline
$i$ & $r_{i-2} = q_{i-1} \cdot r_{i-1}+r_i$ & $r_i = [s_i]r_0 +[t_i]r_1$ \\ 
\hline
$2$ & $911 = 3 \cdot 301 +8$& $r_2=8= [1] 911 + [-3]301$ \\ 
\hline
$3$ & $301 = 37 \cdot 8+5$ & $r_3= 5= 301-37 \cdot 8$ \newline $ r_3 = 301 -37(911-3 \cdot 301)$ \newline $r_3 = -[37]911 + [112]301$\\
\hline
$4$ & $8 = 1 \cdot 5+3$ & $r_4= 3 = 8-5$ \newline $r_4 = 1 \cdot 911 + (-3) \cdot 301 - (-37 \cdot 973 + 112 \cdot 301)$ \newline $r_4 = [38]911 - [115]301$  \\
\hline
$5$ & $5 = 1 \cdot 3+2$ & $r_5= 2= 5-3$ \newline $r_5=-37 \cdot 911 + 112 \cdot 301 - (38 \cdot 911 - 115 \cdot 301)$  \newline  $r_5 = -[75]911 + [227]301$ \\
\hline
$6$ & $3 = 1 \cdot 2+1$ & $r_6= 1 = 3-2$ \newline $r_6=38 \cdot 911 - 115 \cdot 301 -(-75 \cdot 911 +  227 \cdot 301)$ \newline $r_4=[113]911-[342] 301$  \\ 
\hline
$7$ & $2 = 1 \cdot 1+1$ & $r_7= 1 = 2-1$ \newline $r_7=-75 \cdot 911 + 227 \cdot 301 -(113 \cdot 911- 342 \cdot 301)$ \newline $r_7=-[188]911+[569] 301$  \\ 
\hline
\end{tabular}
\end{center}

\noindent
From the EEA we have now computed the inverse of $301$ modulo $911$ as $569$  from the following equation  $1 = -188 \cdot 911 + 569 \cdot 301$.\\

\noindent
To understand how the EEA works we observe that the righthand side is always constructed with the help of the previous linear combinations. We will now derive recursive formulae for computing $s_i$ and $r_i$ in every iteration. Assume we are in iteration with index i.The two previous iterations we computed the values.

\noindent
\begin{alignat*}{2}
r_{i - 2} &= [s_{i-2}]r_0 +[t_{i-2}]r_1\\
r_{i-1} &= [s_{i-1}]r_0 +[t_{i-1}]r_1   
\end{alignat*}



\noindent
In the current iteration $i$ we first compute the quotient $q_{i-1}$ and the new remainder $r_i$ from $r_{i-1}$ and $r_{i-2}$:

\noindent
\begin{alignat*}{2}
r_{i-2} &= q_{i-1} \cdot r_{i-1}+r_i 
\end{alignat*}

\noindent
This equation can be rewritten as:\\

\noindent
\begin{alignat*}{2}
r_i &= r_{i-2}-q_{i-1} \cdot r_{i-1} 
\end{alignat*}


\noindent
The goal is to represent the new remainder $r_i$ as a linear combination of $r_0$ and $r_1$ as $r_i = [s_i]r_0 +[t_i]r_1$. The core step for achieving this is by substitute $r_{i-2}$ and $r_{i-1}$ by the following. The general formula is derived by substitute $r_{i-2}$ and $r_{i-1}$ as:


\noindent
\begin{alignat*}{2}
r_i = (s_{i-2}r_0+t_{i-2}r_1)-q_{i-1}(s_{i-1}r_0+t_{i-1}r_1) 
\end{alignat*}


\noindent
If we rearrange the terms we obtain the desired result:

\noindent
\begin{alignat*}{2}
r_i &= [s_{i-2}-q_{i-1}s_{i-1}]r_0 +[t_{i-2}-q_{i-1}t_{i-1}]r_1\\
r_i &= [s_i]r_0 +[t_i]r_1
\end{alignat*}