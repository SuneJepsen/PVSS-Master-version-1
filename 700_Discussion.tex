In this chaper we will discuss and reflect on the main objectives.  



%********************************************************************
\section{Theoretically}
%********************************************************************

The learning curve has been steep. The literature is on a very high mathematical academic level  seen from the background as software developers. We see this is a challenge that a protocol is described for such a narrow audience. Because one needs to have a academic background within the subject of cryptography. Most of the literature on this subject assume background knowledge on the subject. They offen refers to other literature that this protocol is build upon. With this thesis we have tried to break down the protocol such that it is understandable for a broader audience. We have tried to give a section with the mathematical part which is used to understand the basic structure of this protocol. Hereafter we have reconstructed the description of the  protocol in a basic and a detailed section. The basic description will give the reader basic knowledge about the protocol and should provide enough knowledge for a simple implementation of the protocol. Of cause this simple implementation will not fulfill the security requirements for an electronic voting application.






%********************************************************************
\section{Practical}
%********************************************************************
We have spend much time on understanding the protocol and the security for an election. This has been costly for the practical part of this thesis. We have not been able to develop a complete application, but rather a proof of concept, where we have tested the protocol and our architectural strategies. We would like to have spend more time on the practical part so that we could have tested our QAS such as performance and the availability on our application since performance and availability is central for the software architecture for a large scale electronic voting application. It could be interesting to investigate if theoretical assumption about scaling for large input holds in practice. 


%********************************************************************
\subsection{Generating Prime}
%********************************************************************
Generating very large primes is a computationally hard task and thus time consuming. Given the fact
that the prime $q$ used in our implementation is public known, it is easy to think that one could simply
use one or a set of very large hardcoded primes. However, by performing a large precomputation for a given
prime an adversary can quickly calculate arbitrary discrete logarithms in that group, efficiently reducing 
computation cost for all targets that uses this group \cite{Adrian:2015:IFS:2810103.2813707}. In addition with our work with large numbers it can be a challenge to debug large numbers when the result doesent match the excepted outcome.

%********************************************************************
\subsection{Same-origin security policy}
%********************************************************************
Access-Control-Allow-Origin
- CORS



%********************************************************************
\subsection{Individual Verifiability} \label{sec:discussion_individual_verifiability}
%********************************************************************
Here we will elaborate on these two informal requirements from section \ref{sec:analyzing_electronic_voting_secure_requirements} regarding \textit{Individual Verifiability}. 

\begin{enumerate}
    \item All votes from the ballot casting is included in the final tally.
    \item Every votes from the ballot casting is calculated correctly.
\end{enumerate}

\noindent
As stated earlier our job is to convince the voter about the following statement. If this is possible, we mean that we have argumented for our statements and thereby the security requirement. To elaborate these requirements we will need some illustration of the ballots and the tallying phase. Figure \ref{fig:discussion_individual_verifiability_ballots} shows a simplified list of all valid ballots from the ballot casting phase. Figure \ref{fig:discussion_individual_verifiability_tallying_phase} shows a simplified list of the multiplied shares and the corresponding proofs by the each Tally.\\

\noindent
As for the first statement we refer that all participants can verify the correctness of the votes and the consistency of each share. By definition all is public and all participants and observers are able to verify the validity of the outcome of the proofs under the election.\\

\noindent
So if the list is trusted an accepted by every participants and observers then with high probability we can say it is correct. To ensure that all ballots are included in the final tally we have to be convinced,  that each of the shares are included in the Tallying phase. Each tally will compute $Y^*$, $S^*$ and a proof based on the shares belonging to them. The key is that every participant and observers are able to confirm the $Y^*$ to ensure that all shares are included in the computation of $Y^*$. As for the second statement all can verify that the decryption of the shares are done correctly. All information needed in order for calculating the final tally is publicly available on the Bulletinboard and thus every participant and observer are able to recalculate the final tally.      

\begin{figure}[H]
    \centering
    \captionsetup[subfigure]{labelformat=empty}
    \begin{subfigure}[b]{0.45\textwidth}
        \begin{table}[H]
            \centering
            \begin{tabular}{|l|l|}
                \hline
                \multicolumn{2}{|l|}{Ballot casting} \\ \hline
                Ballot   & Valid              \\ \hline
                1        & Yes                \\ \hline
                2        & Yes                \\ \hline
                3        & Yes                \\ \hline
                4        & Yes                \\ \hline
                5        & Yes                \\ \hline
                6        & Yes                \\ \hline
            \end{tabular}
        \end{table}
        \caption{Ballots with proofs validation}
    \end{subfigure}
    \qquad % <----------------- SPACE BETWEEN PICTURES
    \qquad % <----------------- SPACE BETWEEN PICTURES
    \begin{subfigure}[b]{0.40\textwidth}
        \begin{table}[H]
            \centering
            \begin{tabular}{|l|l|l|l|}
                \hline
                \multicolumn{4}{|l|}{Tallying phase}               \\ \hline
                Tally & Y* & S* & Valid                    \\ \hline
                1     & 10 & 3  & Yes                      \\ \hline
                2     & 8  & 9  & Yes                      \\ \hline
                3     & 5  & 8  & Yes                      \\ \hline
                4     & 9  & 4  & Yes                      \\ \hline
                5     & 9  & 3  & Yes                      \\ \hline
                6     & 2  & 8  & Yes                      \\ \hline
            \end{tabular}
        \end{table}
        \caption{Tallying phase}
    \end{subfigure}
    \caption{Tables showing the states of a Ballots and Tally shares after each phase in a election}
    \label{fig:discussion:individual_verifiability}
\end{figure}







