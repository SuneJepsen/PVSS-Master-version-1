\section{Architectural requirements}






\subsection{Quality Attributes workshop}
In this section we will present 8 phases of the QAW. Despite that a workshop is not held we still see clear benefits by using this model namely to determine the qualities for the electronic voting application before it is implemented. We are well aware that the outcome isn't perfect. 
\begin{description}
    \item [QAW Presentation and Introductions]
        QAW facilitators describe the motivation for the QAW and explain each step of the method. 
    
    \item [Business/Mission Presentation]    
        A representative of the stakeholder community presents the business and/or programmatic drivers for the system.
    
    \item [Architectural Plan Presentation]
        A technical stakeholder presents the system architectural plans as they stand with respect to early documents, such as high-level system descriptions, context drawings, or other artifacts that describe some of the system's technical details.
    
    \item [Identification of Architectural Drivers]
     Architectural drivers often include high-level requirements, business/mission concerns, goals and objectives, and various quality attributes. During this step, the facilitators and stakeholders reach a consensus about which drivers are key to the system.
          
    
    \item [Scenario Brainstorm]
        Stakeholders generate real-world scenarios for the system. Scenarios comprise a related stimulus, an environmental condition, and a response. Facilitators ensure that at least one scenario addresses each of the architectural drivers identified in Step 4.
         
    \item [Consolidation] 
        Scenarios that are similar in content are consolidated.
    
    \item [Prioritization] 
  Stakeholders prioritize the scenarios through a voting process. 
    
    \item [Refinement]
        The top four or five scenarios are further clarified and the following are described:
        \begin{enumerate}
            \item the business/programmatic goals that are affected by those scenarios
            \item the relevant quality attributes associated with those scenarios   
        \end{enumerate}

   

\end{description}




\subsubsection{2. Business / mission presentation}
In this phase we will present our case.
 

\subsubsection{4. Identify architectural drivers}
The purpose of this step is to identify architectural drivers. Architectural drivers are the keys to realizing quality attribute goals for the system. The architectural drivers are often found through requirements and business goals. We found architectural drivers by reflection and discussion against the requirements of earlier described electronic voting application. Based on our architectural drivers, we begin to see the following quality attributes.\\


\begin{table}[H]
\centering
\begin{tabular}{|l|l|}
\hline
\multicolumn{2}{|l|}{Voters point of view}                \\ \hline
Usability    & The system should be easy to use for a voter                   \\ \hline
Usability    & \begin{tabular}[c]{@{}l@{}}The system should be \\ trustworthy and should \\ give detailed feedback \end{tabular}   \\ \hline 
Availability & The system should be available when needed                     \\ \hline
Interoperability         & \begin{tabular}[c]{@{}l@{}} A voter should be able to cast a vote \\ from a given device with a internet \\ connection\end{tabular}                                                                   \\ \hline

\end{tabular}
\caption{Voters point of view}
\label{my-label}
\end{table}


\begin{table}[H]
\centering
\begin{tabular}{|l|l|}
\hline
\multicolumn{2}{|l|}{Robustness}                                                                                                                                                                        \\ \hline
Performence         & \begin{tabular}[c]{@{}l@{}}Should be able to handle a large amount \\ of user within a reasonabletime\end{tabular}                                                                   \\ \hline
Testability      & \begin{tabular}[c]{@{}l@{}}Given the nature of systems complexity it\\ should be easily testableto ensure \\ robustness and reliability\end{tabular}                                 \\ \hline
Security         & \begin{tabular}[c]{@{}l@{}}The integrity should withhold even though \\ if cheating occurs.\end{tabular}                                                                             \\ \hline
\multicolumn{2}{|l|}{Universal verifiability}                                                                                                                                                           \\ \hline
Interoperability & \begin{tabular}[c]{@{}l@{}}Not only participant but also passive observers \\ should be able to validate through out the \\ election and afterwards(Fairness)\end{tabular}           \\ \hline

\multicolumn{2}{|l|}{Future proof}                                                                                                                                                                      \\ \hline
Modifiable       & \begin{tabular}[c]{@{}l@{}}Only a registered user should be able to vote. \\ This registration should be easily replaceable \\ depending on the nature of the election.\end{tabular} \\ \hline
Modifiable       & \begin{tabular}[c]{@{}l@{}}The system should modifiable such that\\  core elements are replaceable\end{tabular}                                                                      \\ \hline
\end{tabular}
\caption{Owners point of view}
\label{my-label}
\end{table}


\subsubsection{5. Scenario brainstorming}
The purpose of this step is to design quality attribute scenarios (QAS), based on our AD's. That is, here we form QAS in a form such as Bass et al. Form so there is clear stimulus and clear response measure.

\begin{table}[H]
\centering

\label{my-label}
\begin{tabular}{|l|l|l|}
\hline
Scenario \#      & Description                                                                                                                                                                                                                                                                          & Votes \\ \hline
Usability        & \begin{tabular}[c]{@{}l@{}}A user casts a vote under \\ runtime and the PVSS client registers\\ the vote with a confirm message, within\\ 5 seconds.\end{tabular}                                                                                                                    &       \\ \hline
Availability     & \begin{tabular}[c]{@{}l@{}}An internal crash occurs and the\\ Bulletin board is out of reach\\ during normal operation. The response\\ is that the error is logged and the \\ system is running in degraded mode.\\ The system should be up running\\ within 5 minutes.\end{tabular} &       \\ \hline
Interoperability & \begin{tabular}[c]{@{}l@{}}A PVSS client cast a vote to the\\ Bulletin board from a given device\\ with internet connection  under runtime and\\ the system is updated and 100\% \\ of the information is exchanged \\ and processed correctly.\end{tabular}                                                                          &       \\ \hline
Interoperability & \begin{tabular}[c]{@{}l@{}}An observer client validates\\ a vote from the the Bulletin board \\ under runtime. The validation is \\ processed, and the Bulletin board is\\ updated and 100\% of the information \\ is exchanged and processed correctly.\end{tabular} &       \\ \hline
Testability      & \begin{tabular}[c]{@{}l@{}}An unitester should be able to\\ code a unit on the system\\ under development and the test suite\\ are executed  and result are captured\\ and 85\% of the system are coverage \\ within 3 hours.\end{tabular} &       \\ \hline
Modifiable       & \begin{tabular}[c]{@{}l@{}}A developer should be able to make a \\ change to the random number generator\\ code under runtime and the change are made\\ and tested within 3 hours.\end{tabular}                                                                                          &       \\ \hline
Modifiable       & \begin{tabular}[c]{@{}l@{}}A developer should be able to make a\\ change to the registration code under \\ runtime and the change are made and tested\\ within 3 hours.\end{tabular}                                                                                                 &       \\ \hline
Security         & \begin{tabular}[c]{@{}l@{}}A cheater cast a invalid vote to the Bulletin\\ board under normal operation. All valid data\\ should be preserved and the system\\ should be able to detect invalid from \\ valid data before the total counting of \\ the votes.\end{tabular}           &       \\ \hline
Performence      & \begin{tabular}[c]{@{}l@{}}1000 transactions are initiates towards \\ the bulletin board under normal operation.\\ The transaction are processed and there are\\ a max latency of 5 seconds.\end{tabular}                                                                            &       \\ \hline
\end{tabular}
\caption{First step towards quality attributes scenarios}
\end{table}

\subsubsection{6. Scenario consideration}
The purpose of this step is to merge scenarios that have similar features. At the workshop there were no scenarios that could be merged.

\subsubsection{7. Scenario prioritization}
The purpose of this step is to draw up a priority list based on the total votes. The list is long but we will limit this thesis to focusing on a few and the rest we will comment. Since the PVSS is a public verifiable protocol, there will be a high focus on fulfilling the "public" requirement. To achieve that we will create web application which can be access by everyone how has a device connected to the internet.

\subsubsection{8. Scenario Refinement}
The purpose of this step is to form QAS based on Bass et al, where divide into source, stimulus, artifact, environment, response and response measure. 



\begin{table}[H]
\begin{center}
\begin{tabular}{|p{0.3cm}|p{2.5cm}|p{8cm}|}
  \hline
  \multicolumn{2}{|p{3cm}|}{\bfseries Scenario(s):} & A secure error was found in the Random generator implementation. The generator needs to be replaced \\
  \hline
  \multicolumn{2}{|p{3cm}|}{\bfseries Relevant Quality Attributes:} & Modifiability\\
  \hline
  \multirow{6}{*}{\begin{sideways}{\bfseries Scenario Parts}\end{sideways}}
  & {\bfseries Source:} & Developer \\
  \cline{2-3}
  & {\bfseries Stimulus:} & Needs to replace the random generator \\
  \cline{2-3}
  & {\bfseries Artifact} &  Code \\
  \cline{2-3}
  & {\bfseries Environment:} &  Design time \\
  \cline{2-3}
  & {\bfseries Response:} &  Replacement made and Unit tested\\
  \cline{2-3}
  & {\bfseries Response Measure:} & In three hours\\
  \hline
  \multicolumn{2}{|p{3cm}|}{\bfseries Questions:} &  \\
  \hline
  \multicolumn{2}{|p{3cm}|}{\bfseries Issues:} &  \\
  \hline
\end{tabular}
\caption{Quality Attribute Scenario}
\end{center}
\end{table}

\begin{table}[H]
\begin{center}
\begin{tabular}{|p{0.3cm}|p{2.5cm}|p{8cm}|}
  \hline
  \multicolumn{2}{|p{3cm}|}{\bfseries Scenario(s):} & A secure error was found in the Random generator implementation. The generator needs to be replaced \\
  \hline
  \multicolumn{2}{|p{3cm}|}{\bfseries Relevant Quality Attributes:} & Modifiability\\
  \hline
  \multirow{6}{*}{\begin{sideways}{\bfseries Scenario Parts}\end{sideways}}
  & {\bfseries Source:} & Developer \\
  \cline{2-3}
  & {\bfseries Stimulus:} & Needs to replace the registration module \\
  \cline{2-3}
  & {\bfseries Artifact} &  Code \\
  \cline{2-3}
  & {\bfseries Environment:} &  Design time \\
  \cline{2-3}
  & {\bfseries Response:} &  Replacement made and Unit tested\\
  \cline{2-3}
  & {\bfseries Response Measure:} & In three hours\\
  \hline
  \multicolumn{2}{|p{3cm}|}{\bfseries Questions:} &  \\
  \hline
  \multicolumn{2}{|p{3cm}|}{\bfseries Issues:} &  \\
  \hline
\end{tabular}
\caption{Quality Attribute Scenario}
\end{center}
\end{table}



\begin{table}[H]
\begin{center}
\begin{tabular}{|p{0.3cm}|p{2.5cm}|p{8cm}|}
  \hline
  \multicolumn{2}{|p{3cm}|}{\bfseries Scenario(s):} & \\
  \hline
  \multicolumn{2}{|p{3cm}|}{\bfseries Relevant Quality Attributes:} & Interoperability\\
  \hline
  \multirow{6}{*}{\begin{sideways}{\bfseries Scenario Parts}\end{sideways}}
  & {\bfseries Source:} & A webbrowser \\
  \cline{2-3}
  & {\bfseries Stimulus:} & Validate a vote \\
  \cline{2-3}
  & {\bfseries Artifact} &  Bulletinboard system \\
  \cline{2-3}
  & {\bfseries Environment:} &  Runtime \\
  \cline{2-3}
  & {\bfseries Response:} &  If the vote is valid it is accepted. If the vote is invalid it is rejected and the vote is removed from the Bulletinboard \\
  \cline{2-3}
  & {\bfseries Response Measure:} & 100 \% of the information is exchange and processed correctly \\
  \hline
  \multicolumn{2}{|p{3cm}|}{\bfseries Questions:} &  \\
  \hline
  \multicolumn{2}{|p{3cm}|}{\bfseries Issues:} &  \\
  \hline
\end{tabular}
\caption{Quality Attribute Scenario}
\end{center}
\end{table}


\begin{table}[H]
\begin{center}
\begin{tabular}{|p{0.3cm}|p{2.5cm}|p{8cm}|}
  \hline
  \multicolumn{2}{|p{3cm}|}{\bfseries Scenario(s):} & \\
  \hline
  \multicolumn{2}{|p{3cm}|}{\bfseries Relevant Quality Attributes:} & Security\\
  \hline
  \multirow{6}{*}{\begin{sideways}{\bfseries Scenario Parts}\end{sideways}}
  & {\bfseries Source:} & A webbrowser \\
  \cline{2-3}
  & {\bfseries Stimulus:} & Validate a vote \\
  \cline{2-3}
  & {\bfseries Artifact} &  Bulletinboard system \\
  \cline{2-3}
  & {\bfseries Environment:} &  Runtime \\
  \cline{2-3}
  & {\bfseries Response:} &  If the vote is valid it is accepted. If the vote is invalid it is rejected and the vote is removed from the Bulletinboard \\
  \cline{2-3}
  & {\bfseries Response Measure:} & 100 \% of the information is exchange and processed correctly \\
  \hline
  \multicolumn{2}{|p{3cm}|}{\bfseries Questions:} &  \\
  \hline
  \multicolumn{2}{|p{3cm}|}{\bfseries Issues:} &  \\
  \hline
\end{tabular}
\caption{Quality Attribute Scenario}
\end{center}
\end{table}



\subsection{Architectural decision}


\subsection{Architectural evaluation}