\section{Architectural requirements}






\subsection{Quality Attributes workshop}
In this section we will present 8 phases of the QAW. Despite that a workshop is not held we still see clear benefits by using this model namely to determine the qualities for the electronic voting application before it is implemented. We are well aware that the outcome isn't perfect. 
\begin{description}
    \item [QAW Presentation and Introductions]
        QAW facilitators describe the motivation for the QAW and explain each step of the method. 
    
    \item [Business/Mission Presentation]    
        A representative of the stakeholder community presents the business and/or programmatic drivers for the system.
    
    \item [Architectural Plan Presentation]
        A technical stakeholder presents the system architectural plans as they stand with respect to early documents, such as high-level system descriptions, context drawings, or other artifacts that describe some of the system's technical details.
    
    \item [Identification of Architectural Drivers]
     Architectural drivers often include high-level requirements, business/mission concerns, goals and objectives, and various quality attributes. During this step, the facilitators and stakeholders reach a consensus about which drivers are key to the system.
          
    
    \item [Scenario Brainstorm]
        Stakeholders generate real-world scenarios for the system. Scenarios comprise a related stimulus, an environmental condition, and a response. Facilitators ensure that at least one scenario addresses each of the architectural drivers identified in Step 4.
         
    \item [Consolidation] 
        Scenarios that are similar in content are consolidated.
    
    \item [Prioritization] 
  Stakeholders prioritize the scenarios through a voting process. 
    
    \item [Refinement]
        The top four or five scenarios are further clarified and the following are described:
        \begin{enumerate}
            \item the business/programmatic goals that are affected by those scenarios
            \item the relevant quality attributes associated with those scenarios   
        \end{enumerate}

   

\end{description}




\subsubsection{2. Business / mission presentation}
In this phase we will present our case.
 

\subsubsection{4. Identify architectural drivers}
The purpose of this step is to identify architectural drivers (AD). We found AD's by discussions while we got them described on the board. Based on our drivers, we could begin to see the following quality attributes.\\


\begin{table}[H]
\centering
\begin{tabular}{|l|l|}
\hline
\multicolumn{2}{|l|}{Voters point of view}                \\ \hline
Usability    & Easy to use for a voter                   \\ \hline
Usability    & Trustworthy should give detailed feedback \\ \hline
Availability & Available when needed                     \\ \hline
\end{tabular}
\caption{Voters point of view}
\label{my-label}
\end{table}


\begin{table}[H]
\centering
\begin{tabular}{|l|l|}
\hline
\multicolumn{2}{|l|}{Robustness}                                                                                                                                                                        \\ \hline
Scalable         & \begin{tabular}[c]{@{}l@{}}Should be able to handle a large amount \\ of user within a reasonabletime\end{tabular}                                                                   \\ \hline
Testability      & \begin{tabular}[c]{@{}l@{}}Given the nature of systems complexity it\\ should be easily testableto ensure \\ robustness and reliability\end{tabular}                                 \\ \hline
Security         & \begin{tabular}[c]{@{}l@{}}The integrity should withhold even though \\ if cheating occurs.\end{tabular}                                                                             \\ \hline
\multicolumn{2}{|l|}{Universal verifiability}                                                                                                                                                           \\ \hline
Interoperability & \begin{tabular}[c]{@{}l@{}}Not only participant but also passive observers \\ should be able to validate through out the \\ election and afterwards(Fairness)\end{tabular}           \\ \hline
\multicolumn{2}{|l|}{Future proof}                                                                                                                                                                      \\ \hline
Modifiable       & \begin{tabular}[c]{@{}l@{}}Only a registered user should be able to vote. \\ This registration should be easily replaceable \\ depending on the nature of the election.\end{tabular} \\ \hline
Modifiable       & \begin{tabular}[c]{@{}l@{}}The system should modifiable such that\\  core elements are replaceable\end{tabular}                                                                      \\ \hline
\end{tabular}
\caption{Owners point of view}
\label{my-label}
\end{table}


\subsubsection{5. Scenario brainstorming}
The purpose of this step is to design quality attribute scenarios (QAS), based on our AD's. That is, here we form QAS in a form such as Bass et al. Form so there is clear stimulus and clear response measure.


\subsubsection{6. Scenario consideration}
The purpose of this step is to merge scenarios that have similar features. At the workshop there were no scenarios that could be merged.

\subsubsection{7. Scenario prioritization}
The purpose of this step is to draw up a priority list based on the total votes.

\subsubsection{8. Scenario Refinement}
The purpose of this step is to form QAS based on Bass et al, where divide into source, stimulus, artifact, environment, response and response measure. 



\begin{table}[H]
\begin{center}
\begin{tabular}{|p{0.3cm}|p{2.5cm}|p{8cm}|}
  \hline
  \multicolumn{2}{|p{3cm}|}{\bfseries Scenario(s):} & A secure error was found in the Random generator implementation. The generator needs to be replaced \\
  \hline
  \multicolumn{2}{|p{3cm}|}{\bfseries Relevant Quality Attributes:} & Modifiability\\
  \hline
  \multirow{6}{*}{\begin{sideways}{\bfseries Scenario Parts}\end{sideways}}
  & {\bfseries Source:} & Developer \\
  \cline{2-3}
  & {\bfseries Stimulus:} & Needs to replace the random generator \\
  \cline{2-3}
  & {\bfseries Artifact} &  Code \\
  \cline{2-3}
  & {\bfseries Environment:} &  Design time \\
  \cline{2-3}
  & {\bfseries Response:} &  Replacement made and Unit tested\\
  \cline{2-3}
  & {\bfseries Response Measure:} & In one hour\\
  \hline
  \multicolumn{2}{|p{3cm}|}{\bfseries Questions:} &  \\
  \hline
  \multicolumn{2}{|p{3cm}|}{\bfseries Issues:} &  \\
  \hline
\end{tabular}
\caption{Quality Attribute Scenario}
\end{center}
\end{table}

\begin{table}[H]
\begin{center}
\begin{tabular}{|p{0.3cm}|p{2.5cm}|p{8cm}|}
  \hline
  \multicolumn{2}{|p{3cm}|}{\bfseries Scenario(s):} & A secure error was found in the Random generator implementation. The generator needs to be replaced \\
  \hline
  \multicolumn{2}{|p{3cm}|}{\bfseries Relevant Quality Attributes:} & Modifiability\\
  \hline
  \multirow{6}{*}{\begin{sideways}{\bfseries Scenario Parts}\end{sideways}}
  & {\bfseries Source:} & Developer \\
  \cline{2-3}
  & {\bfseries Stimulus:} & Needs to replace the random generator \\
  \cline{2-3}
  & {\bfseries Artifact} &  Code \\
  \cline{2-3}
  & {\bfseries Environment:} &  Design time \\
  \cline{2-3}
  & {\bfseries Response:} &  Replacement made and Unit tested\\
  \cline{2-3}
  & {\bfseries Response Measure:} & In one hour\\
  \hline
  \multicolumn{2}{|p{3cm}|}{\bfseries Questions:} &  \\
  \hline
  \multicolumn{2}{|p{3cm}|}{\bfseries Issues:} &  \\
  \hline
\end{tabular}
\caption{Quality Attribute Scenario}
\end{center}
\end{table}


\subsection{Architectural decision}


\subsection{Architectural evaluation}