\section{Architectural requirements}    
\subsection{Quality Attributes workshop}

\begin{table}[!htp]
\begin{center}
\begin{tabular}{|p{0.3cm}|p{2.5cm}|p{8cm}|}
  \hline
  \multicolumn{2}{|p{3cm}|}{\bfseries Scenario(s):} & A secure error was found in the Random generator implementation. The generator needs to be replaced \\
  \hline
  \multicolumn{2}{|p{3cm}|}{\bfseries Relevant Quality Attributes:} & Modifiability\\
  \hline
  \multirow{6}{*}{\begin{sideways}{\bfseries Scenario Parts}\end{sideways}}
  & {\bfseries Source:} & Developer \\
  \cline{2-3}
  & {\bfseries Stimulus:} & Needs to replace the random generator \\
  \cline{2-3}
  & {\bfseries Artifact} &  Code \\
  \cline{2-3}
  & {\bfseries Environment:} &  Design time \\
  \cline{2-3}
  & {\bfseries Response:} &  Replacement made and Unit tested\\
  \cline{2-3}
  & {\bfseries Response Measure:} & In one hour\\
  \hline
  \multicolumn{2}{|p{3cm}|}{\bfseries Questions:} &  \\
  \hline
  \multicolumn{2}{|p{3cm}|}{\bfseries Issues:} &  \\
  \hline
\end{tabular}
\caption{Quality Attribute Scenario}
\end{center}
\end{table}

\subsubsection{2. Business / mission presentation}

På baggrund af den brede formulering af TM16 har vi valgt at udforme en mission præsentation af TM16, med vores vision for TM16. Mission Presentation danner grundlag for QA workshoppen og med en for bred specifikation af TM16 vil vores scenarie kunne misforstås. 


Vi har valgt at anse TM16 som et redskab i et screen forløb af en patient. Når en patient indgår i et screen forløb stilles en måler med TM16 til rådighed. Patienten skal så over en given periode løbende fortage målinger. Når perioden er gået, har fagpersonale mulighed for at inddrage målingerne i evalueringen af patienten over den givne periode. 


Det skal være muligt at benytte TM16 uden internet adgang, da vi ikke kan antage at alle patienter har internet adgang i hjemmet. I tilfælde af at der ikke har været internet adgang under en måling, så sendes denne måling til en intern database på måleenheden. Måliningen sendes til TM16 Server så snart etableres en internet forbindelse.



TM16 skal væres nem at betjene da patienterne har meget forskellige faglige baggrund, hvor vi må antage at nogle af patienterne har meget ringe eller ingen teknisk færdigheder. 

Det skal desuden være muligt for fagpersonalet at arbejde med målingerne fra TM16 i deres eget fagprogram dvs. et specifik program til specifikke faggrupper. Det forventes at målingerne kan trækkes direkte fra TM16 så snart at målingerne at sendt til TM16 server. Da TM16 forventes at blive en kæmpe succes så vil der komme stor tilgang af brugere og sandsynligheden for samtidighed vil dermed opstå. Derfor skal systemet kunne håndter samtighed. Vi formoder at i takt med at flere faggrupper får adgang til TM16, så vil der blive stilet krav til flere forskellige typer målingerne.

\subsubsection{4. Identify architural drivers}
Voter point of view: \\
Usability: Easy to use for a voter\\
Usability: Trustworthy should give detailed feedback\\
Availability: available when needed\\\\


\noindent
Owner point of view:\\

\noindent
Robustness:\\
Scalable: Should be able to handle a large amount of user within a reasonable time\\
Testability: Given the nature of systems complexity it should be easily testable to ensure robustness and reliability\\
Security: The integrity should withhold even though if cheating occurs.\\

\noindent
Universal verifiability\\
Interoperability:Not only participant but also passive observers should be able to validate through out the election and afterwards(Fairness)\\


\noindent
Future proof\\
Modifiable: only a registered user should be able to vote. This registration should be easily replaceable depending on the nature of the election.\\
Modifiable: The system should modifiable such that core elements are replaceable\\








\subsubsection{5. Scenario brainstorming}

\subsubsection{6. Scenario considation}

\subsubsection{7. Scenario prioritization}

\subsubsection{8. Scenario Refinement}
        