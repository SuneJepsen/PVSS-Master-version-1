\section{Cryptographic tools}
The following section will be about the cryptographic assumptions which are used in the PVSS protocol. We will also describe some techniques which will be used in practical part of coding the protocol.


%------------------------------------------------------------------------------------
\subsection{Zero knowledge proof} % Zero knowledge proof
%------------------------------------------------------------------------------------
In cryptography, a zero-knowledge proof is a protocol by which one party (the prover) can prove to another party (the verifier) that a given statement is true, without revealing any information apart from the fact that the statement is indeed true.\\

\noindent
Zero knowledge proof is a important part of the PVSS protocol. It is used for verifying the correctness of the data. Intuitively Zero knowledge proof is a protocol between two parties a prover and a verifier where the prover tries to convince the verifier about some statement of the form $x \in L$, for which he uses additional knowledge about $x$. In the last step if the protocol, the verifier either accepts or rejects the proof. A zero-knowledge proof must satisfy following three properties:

%-----------------------------------------------zero knowledge
\begin{itemize}
\item  \textnormal{\textbf{(Completeness:)}} If the prover is honest and the statement is true, then the honest verifier always accept. More formally it means that $x \in L$.
\item    \textnormal{\textbf{(Soundness:)}} If the statement is false then it should fail with overwhelming probability. More formally it means that $x \not\in L$.
\item   \textnormal{\textbf{(Zero-knowledge:)}} If the statement is true, no cheating verifier learns anything other than the fact that the statement is true.
\end{itemize}
%-----------------------------------------------zero knowledge


%------------------------------------------------------------------------------------
\subsection{Discrete Logarithm problem} % and One way functions
%------------------------------------------------------------------------------------

\parahead{One-way function} One-way functions are easy to compute but it is very difficult to compute their inverse functions. Thus, having data $x$ it is easy to calculate $f(x)$ but,  
knowing only the result of $f(x)$ it is hard to calculate the value of $x$. We say a function $f(x)$ is easy to compute if this can be done in polynomial running time. 
In order to be useful in practical crypto schemes, the computation of $f(x)$ should be fast enough that it does not lead to unacceptably slow execution times in an application. The inverse
computation of $f(x)$ should be so computationally intensive that it is not feasible to evaluate it in any reasonable time period. We define a One-way function as \\

\begin{defi}[One-way function]
A function $f()$ is a one-way function if:      \\
1. $y = f (x)$ is computationally easy           \\
2. $x = f^{-1}(y)$ is computationally infeasible    
\end{defi}

\noindent
An example of a one-way function is $n = pq$ where $p$ and $q$ are primes, it is easy to compute $n$ given $p$ and $q$ but hard to find $p$ and $q$ given only $n$. 
Inverting this function requires finding the factors of $n$.
An other example of a one-way function is Discrete logarithm. \\


\parahead{Discrete logarithm (DL)} A discrete logarithm is a integer $a$ exponent that solves $g^a=c$, where $g$ is a generator and $c$ is a element of a cyclic group. Given $a$ its easy to compute, but given only $g$ and $c$ its very hard to find $a$. We define discrete logarithm as \\

\begin{defi}[Discrete logarithm (DL) problem]
Given a group $G$, generator $g$ and $c \in G$, find integer $a$, such that $g^a = c$
\end{defi}

\noindent
An example of the DL problem could be the following. Given a group $G = \Z_{47}^*$, an generator $g=5$, and an element $c = 41$ find a integer $a$ to solve: 

\begin{center}
$
\begin{array}{l}
     5^a \stackrel{?}{=} 41 \ mod \ 47 \\
     \\
     \text{To solve this DL problem we need to find} \\
     a = 15 \ as \ 5^{15} = 41 \ mod \ 47
\end{array}
$
\end{center}

\noindent
To solve this DL problem we could just tried all possible solution of \\
$ \{g^0, g^1, g^2,...,g^{46}\} \ mod \ 47$ until we find the correct answer. However it is easy to see that given a large enough group this would be ineffective, actually the DL problem is believed to be notoriously hard, for instance in $\Z_p^*$ for large prime $p$. \\

\noindent
We use the DL problem in the following

% Comment out since Kl dont know if we need this. 
\iffalse
    \begin{defi}[Computational Diffie-Hellman (CDH) problem]
    \begin{math}g\in\Z_p, \ g\neq1 \end{math}\\
    Given \begin{math}(g,g^a,g^b)\end{math} find(compute)  \begin{math}(g^{a*b})\end{math} is hard problem.\\
    Definition: Need a reference?? \\
    \textcolor{red}{Kasper}
    \end{defi}
\fi

\parahead{Diffie-Hellman problem (DHP)}
One of the best known application of the DL problem is in the Diffie-Hellman problem an in particular in the Diffie-Hellman key exchange. Though the Diffie-Hellman key exchange is not used in the PVSS protocol we will use it to illustrate the DHP.


\begin{figure}[H]
    \centering        
    
    $
    \begin{array}{l}
    \hline                      \
    \textbf{Diffie-Hellman Key Exchange}      \\
    \hline                      \\
    \text{Public: a group G and a generator g}      \\
    \\
	\begin{array}{L{1.1cm}lcl}
        & \text{\textsf{Alice}} & \text{\textsf{Eve}}_{Adversery} & \text{\textsf{Bob}} \\
        \hline
        Step \ 1    &           \begin{array}{l}
                                   Choses: \ a \in_R G        \\ 
                                   Compute: \ A = g^a         
                                \end{array}     & \xrightarrow{\hspace{3em} A \hspace{3em}}  & \\
        Step \ 2    &                           & \xleftarrow{\hspace{3em} B \hspace{3em}}       &                                                    \begin{array}{l}
                                                            Choses: \ b \in_R G \\
                                                            Compute: \ B = g^b
                                                        \end{array} \\
                    \\
        Step \ 3    &   \begin{array}{ll} 
                            C &= (B)^a      \\
                              &= (g^b)^a    \\
                              &= g^{ab}
                        \end{array}   &     \xleftrightarrow{\hspace{0.5em}encrypt_C(message)\hspace{0.5em}} & \begin{array}{ll} 
                            C &= (A)^b      \\
                              &= (g^a)^b    \\
                              &= g^{ab}     
                        \end{array}     \\                                                 
        \hline
    \end{array}
    \end{array}
    $    
    \caption{Diffie-Hellman Key Exchange}
	\label{fig:Diffie_Hellman_KeyExchange}
\end{figure}

\noindent
As shown in step 1 Alice is independently from Bob choosing an random element from the group G and using this to create a DL problem A which is sent to Bob. In step 2 Bob is doing the same procedure as Alice and sends a DL problem B to Bob. In step 3 they both independently from each other using the received DL problems to create a key C. Both Alice and Bob will compute the same value C which can be used as a key in a cryptographic scheme. \\

\noindent
If we look at the exchange from Eve's point of view, we can see that Eve knows the public elements G and g aswell as A and B from step 1 and 2. If Eve can compute $C = g^{ab}$ then Eve would be able to decrypt any message sent between Alice and Bob. We define this problem as. \\

\begin{defi}[Computational Diffie-Hellman (CDH) problem]
Given a group $G$, generator $g$ and $A = g^a$, $B = g^b$, where $a,b$ is are randomly independently chosen from $\Zp$, compute $C=g^{ab}$ 
\end{defi}

\noindent
If Eve knows an efficient algorithm to solve the DL problem, then Eve would also be able to solve the CDH problem. Finding $a$ from $A = g^a$ or $b$ from $B = g^b$ then Eve can easily compute $C$ the same way that Alice and Bob was able to. which leads us to  

\begin{lemma}
The CDH problem is no harder then the DL problem
\end{lemma}

\noindent
It is not known if the opposite direction is true in general, but in some groups, the problems are equivalent.      

\parahead{Decisional Diffie-Hellman (DDH) problem} The CDH problem have another property namely if given a group element and the claim that this solves a CDH instance, then is not easy to verify that the solution is correct unless we can solve the CDH problem. 
\noindent
We would need to decide if, given $g^a, g^b, g^c$, if it holds that $c = ab \ mod \ p$. we define this as. \\

\iffalse
    \begin{defi}[Decisional Diffie-Hellman (DDH) problem]
    \begin{math}g\in\Z_p, \ g\neq1 \end{math}\\ 
    Given \begin{math}(g,g^a,g^b,g^c)\end{math} decide if  \begin{math}(a*b=c)\end{math} is hard problem.\\
    Definition: Need a reference??
    \end{defi}
\fi 

\begin{defi}[Decisional Diffie-Hellman (DDH) problem] 
    Given a group $G$, a generator $g$ and $A = g^a$, $B = g^b$ and $C = g^c$, where $a$ and $b$ are randomly and independently chosen from $\Z_p$ and where $c$ is chosen either as $c = ab$ or uniformly random from $\Zp$. Now guess if $c$ is the product of $ab$ or randomly chosen.   
\end{defi}

Looking at the key-exchange protocol again, then if Eve calculate a value C and present this to Alice with the claim that this is a valid $C$, then Alice could easily tell if this is true or false as Alice is able to compute $C = g^{ab}$ and could then just simply compare the two values. However if reversed as in Alice gives Eve a value $C$ and the claim that this is valid, then Eve cannot verify this unless Eve is able to solve the CDH problem, this lead us to.

\begin{lemma}
The DDH problem is no harder then CDH problem
\end{lemma}

\parahead{In the PVSS protocol}  
\textcolor{red}{Needs some more filling :)}

The security of the PVSS protocol can be reduced down to the hardness of the DL problem. Which means that if there can be found a efficient algorithm to solve the DL problem with large expoents then the entire PVSS protocol is insecure. In the next subsection we will look at known algorithms for solving the DL problem. 

%------------------------------------------------------------------------------------
\subsection{Solving the Discrete Logarithm Problem}
%------------------------------------------------------------------------------------
We will here look at some of the known algorithms to solve the DL problem. It is known that the DL problem can be solved given enough time and compute power, so we only ask that this cannot be done within \textcolor{red}{What do we need here?  polynomial time?}. In order to increase the computation time needed to solve the DL problem, we increase the size of the groups we operate within. However the consequence of increasing the group size is an increased execution time of our protocol. \\

\noindent
Below we have listed some of the known algorithms. These algorithms all takes the same input and gives the same output. The input is a cyclic group $G$, an generator $g$ and a element $c \in G$, and we can calculate the $t = ord(G)$. The output is an integer $a$ that solves $g^a = c$

\begin{enumerate}
    \item \textbf{Brute-force / Exhausted search} \\
    This algorithms tries every solution until it finds the correct one, by simply calculating
    $g^0, g^1, g^2, g^3....g^{t-1}$ and then testing if $g^i \stackrel{?}{=} c$.   
    
    \noindent
    This algorithm have a running time of \bigO{n}, as it have to calculate
    every single potential solution. 
    
    \item \textbf{Square-root attacks} \\
    The main idea with square-root attacks is to divide the DL problem into small problems, which can be calculate more efficiently and faster. 
    
    \noindent
    As the name square-root attacks implies, the running time is around \bigO{\sqrt{n}} \\

    \noindent
    There are several known square-root attacks, one of them is the \textbf{[Baby-steps Giant-steps algorithm]} which we will look at in details below. \\          
    
    \item \textbf{Index-Calculus attacks} \\
    Index-Calculus attacks are the best known attack against DL problem, but it only
    works in certain groups, in particularly $\Z_p^* $ and $ GF(2^m)^* $. \\
    
    This means in practice that $p$ needs to between $ 2^{1024} $ to $ 2^{2048}$
\end{enumerate}


%----------------------------------------------------------------
\parahead{Baby-steps Giant-steps algorithm} 
%----------------------------------------------------------------
We will here look in details how the Baby-steps Giant-steps algorithm works. For simplicity and as our implantation of the PVSS protocol operates in $\Zp^*$ we will only look at the algorithm in this group. \\

\noindent
The idea of the algorithm is to divide the group into smaller sub-groups. Given a cyclic group $G$, a generator $g$ and an element $c \in g$, we can then think of all the elements in $G$ as points in a circle 

\begin{center}
    $1 = g^0, g^1, g^2,...,g^{q-2}, g^{q-1}, g^q = 1$
\end{center}

\noindent
We then divide then circle into intervals of $u = \lceil \sqrt{p} \rceil$ sizes, each interval is the Giant step whereas the elements in the intervals is the baby steps which there are at most $u$ of. We can now look at the exponent $a$ as the product of $a = ui + j$ where $0 \leq i,j \leq u$. This allows us to restate the problem as such
\begin{align*}
    c           &= g^a                  \\
    c           &= g^{ui + j}           \\ 
    c           &= g^{ui} \cdot g^{j}   \\
    c( g^{-u} )^i &= g^{j}                
\end{align*}
\noindent
The goal now, is to find an integer $j$ and $i$ such that $c( g^{-u} )^i = g^{j}$. this can be done by computing $g^{j}$ for $j = 0,1,...,u-1$ and $c( g^{-u} )^i = g^{j}$ for  $i = 0,1,...,u-1$ and then finding a match between the two lists. The details are given in algorithm \ref{alg:BabyGiant}

\begin{center}
\begin{algorithm}[H]
\caption{Baby-steps Giant-steps algorithm  \label{alg:BabyGiant}}

\SetKwRepeat{Do}{do}{while}

\KwIn{Group $G$ of order $p$, a generator $g$, an element $c \in G$}
\KwOut{An integer $a$ such that $g^a = c$}

\Begin{
    $u = \lceil \sqrt{p} \rceil$ \\
    \For{$j=0$ to $u-1$}{     
    Compute $g^j \ mod \ p$ and store the pair ($j$,$g^j$) with $g^j$ as key, in a table $t$
    }
    Compute $g^{-u} \ mod \ p$ \\
    \For{$i=0$ to $u-1$}{
    Compute $x = c(g^{-u})^i$
    
    \If{$x$ is in table $t$}{
        \Return{ $a = ui + j$ }
    }    
    }    
}
\end{algorithm}
\end{center}
\noindent
By example we valid the algorithm. Let $p = 31$, $g=3$ and $c = 6$.

\begin{enumerate}
\item $u = \lceil \sqrt{p} \rceil = 6$
\item Computing $1,g,g^1,...,g^5$ gives us  \\

\begin{tabular}{|l|c|c|c|c|c|c|}
    \hline  
    $0 \leq i \leq u-1$             &   $g^0$   &   $g^1$     &   $g^2$     &   $g^3$     &   $g^4$     &   $g^5$     \\
    \hline  
    $g^i \ mod \ p$ &   $1$       &   \textcolor{red}{$3$}       &   $9$       &   $27$      &   $17$      &   $26$      \\
    \hline 
\end{tabular}

\item Next we compute $g^{-u} = 3^{-6}$, we can use the euclidean algorithm to find $g^{-1} = 3^{-1} = 21$. using this result we get $3^{-6} = 21^6 = 2 \ mod \ 31$. 

\item Using the result for the previous step we can efficiently compute $c(g^{-u})^j$ for an increasing $j$ until we find a match with the result from step 2. \\

\iffalse
\begin{align*}
    c(g^{-u})^0 &= 6 \cdot 2^0 = 6 \\
    c(g^{-u})^1 &= 6 \cdot 2^1 = 12 \\
    c(g^{-u})^2 &= 6 \cdot 2^2 = 24 \\
    c(g^{-u})^3 &= 6 \cdot 2^3 = 17 \ mod \ 31 \\
    c(g^{-u})^4 &= 6 \cdot 2^4 = 3 \ mod \ 13
\end{align*}
\fi 

\begin{tabular}{|l|c|c|c|c|c|}
    \hline  
    $0 \leq j \leq u-1$    &   $6 \cdot 2^0$   &   $6 \cdot 2^1$     &   $6 \cdot 2^2$     &   $6 \cdot 2^3$     &   $6 \cdot 2^4$        \\
    \hline  
    $c(g^{-u})^2 = 6 \cdot 2^j \ mod \ 31$ &   $6$   &   $12$   &   $24$   &   $17$      &   \textcolor{red}{$3$}       \\
    \hline 
\end{tabular}

We found a match as $c(g^{-u})^4 = g^1 \ mod \ p$. 
\item We can then compute $a = g^{iu+j} = g^{4u+1} = g^{4 \cdot 6 + 1} = g^{25} = 3^{25} = \underline{\underline{6}} \ mod \ 31$
\end{enumerate}



%------------------------------------------------------------------------------------
\subsection{Hash function}
%------------------------------------------------------------------------------------
Hash functions is a function that takes a message of arbitrary size and outputs a digest hash value of a fixed size. The hash functions used in this project are collision resistant hash function, which means that they are considered as one-way function. we define collision resistant hash function, denote just as hash functions onward, as.

\begin{defi}[Hash function]
\begin{enumerate}
    \item Takes a message of arbitrary size and outputs a value of fixed size
    \item Is deterministic so the same message always results in the same hash
    \item Is quick to compute the hash value for any given message
    \item Is collision resistant, mean it is infeasible to find two inputs $x$ and $x'$ such that $H(x) = H(x')$, and $x \neq x'$
    \item A small change to a message should change the hash value so extensively that the new hash value appears uncorrelated with the old hash value
\end{enumerate}
\end{defi}

A function hash function: $\{0,1\}^{\leq L} \rightarrow \{0,1\}^\ell$ is called collision resistant if it is hard to find $x \in \{0,1\}^{\leq L}$ and $x' \in \{0,1\}^{\leq L}$ such that $x \neq x'$ and $H(x)=H(x')$ - the value $(x,x')$ is called a collision. Here $\{0,1\}^{\leq L}$ denotes the set of bitstrings of length at most $L$. If $L \geq \ell$, then of course collisions exist, so they can be found given enough time, which is fine as we only ask that they are computationally hard to find.  

%------------------------------------------------------------------------------------
\subsection{Fiat Sharmir}
%------------------------------------------------------------------------------------
The Fiat–Shamir heuristic is a technique in cryptography for taking an interactive proof of knowledge into a non interactive proof (signature scheme). This way, some fact (for example, knowledge of a certain number secret to the public) can be proven without revealing underlying information. This means that transforming a interactive proof into a non-interactive proof. Instead of the verifier creates a challenge the prover creates a challenge, on some previous data, based on a hash function.

\iffalse
\noindent
$Gen_{id}$: is the generator of $pk$ and $sk$.

\begin{figure}[H]
    \centering        
    
    $
    \begin{array}{l}
    \hline                      \
    \textbf{Fiat Sharmir}      \\
    \hline                      \\
    \text{Public: identification schemes} \ Gen_{id}, P_1, P_2, V       \\
    \text{The Signer: Private key} \ sk \text{, public key} \ pk \text{, message} \ m \in \{1,0\}  \\
    \\
	\begin{array}{L{1.1cm}lcl}
        & \text{\textsf{Signer}}_{sk,pk, m} & & \text{\textsf{Verifier}} \\
        \hline
        Step \ 1    &           \begin{array}{l}
                                    (I,st) \leftarrow P_1(sk)             \\ 
                                    r := H(I,m)      \\ 
                                    s := P_2(sk,st,r)    
                                \end{array}     &                                   & \\
                    &                           &                                   & \\
        Step \ 2    &                           &       \xrightarrow{\hspace{1em} r, \ s, \ pk \hspace{1em}} & \begin{array}{l}
                                                            Compute: \\ 
                                                            I := V(pk,r,s) \\ \\
                                                            Outputs: \\ 
                                                            1 \ \text{if and only if} \ H(I,m) \stackrel{?}{=} r \\
                                                        \end{array} \\
        \hline
    \end{array}
    \end{array}
    $    
    \caption{Fiat Sharmir}
	\label{fig:Fiat__Sharmir}
\end{figure}
\fi