\section{Case}
The case is as following. A user applies for voting for a given election and a registration authority will either accept or reject his application.  If the user is accepted then he should be able to logon a voting page and cast his vote.  When all the registered voters have casted their votes or a deadline is reached, the system should tally the votes and then publish the result on a webpage. Offcause only valid votes are included in the talliering process. The tally process should be handled by registered talliers. During this process none of the talliers should be able relate a  vote to a voter. Nor should a tallier be able to manipulate the talliering process by either adding, removing or alter votes. Each tallier should be hold accountable for his participant in the talliering process. If the tallier is discovered in cheating he is replaced by another tallier during the tallier process. This replacement should not have influence on the result nor should it required re-election. We use the general security requirement for an electronic voting scheme described in chapter 2 as functional requirements for the application. These requirements are well studied and discussed and should be comprehensive for an electronic voting scheme.