\section{Multiparty Computation}
\subsection{Adversaries}
Participants in a MPC protocol do not necessarily behave as intended. Corrupt
players may occur and they are divided into the following two main
groups [Opp05]:
• A passive adversary is where players perform the protocol correctly,
but collaborate in information gathering and sharing with the goal of
getting access to sensitive information of other players. Such players
are sometimes called semihonest.
• An active adversary is a group of players deviating from the protocol
in order to disrupt the computation. The goal is to produce incorrect
results and/or violate the privacy of other players. Such players are
sometimes called byzantine.
13
Chapter 3. Secure Multiparty Computation
Both types can be static or adaptive. A static adversary means that the
set of corrupted players is chosen before the protocol starts. An adaptive
adversary on the other hand, can choose which players to corrupt during
the execution of the protocol.