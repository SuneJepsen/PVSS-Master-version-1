\section{Multiparty Computation}
This problem was first introduced by Yao in 1982 and exemplified through what is known as the "millionaire problem" \cite{Yao82}:

\begin{center}
\textit{“Two millionaires wish to know who is richer; however, they do not want to find out inadvertently any additional information about each other’s wealth. How can they carry out such a conversation?”}
\end{center}

\noindent
There are two types of MPC protocols. First there are MPC protocols which are secure against passive corruption which assumes that everyone is honest and sends correct data. Then there are MPC protocols which are secure against active corruption where adversary is able to send incorrect data. Here we need more advanced protocols like the Verifiable secret sharing protocol (VSS) and Zero knowledge proofs. Here the protocol allows the involved participants to verify their shares as consistent. This means that the VSS will be able to detect incorrect shares. \\



\noindent
An extension to the VSS protocol is PVSS protocol where the goal is not just that the participants can verify their own shares, but that anybody can verify the correctness of the transmitted data. 

\subsection{Adversaries}
The reason for this is that we cannot rely on every participants in a MPC protocol to behave as intended. Some participants could be corrupted, we divide these adversaries into two main groups \cite{IntroCrypto}. 

\begin{itemize}
\item \textbf{Passive corruption} is that a server (semihonest) gets access to information which the server is not entitled to, e.g. if a voter ask another voter about his information and tries to compare the information and get some more information in that way. By using secret sharing we can prevent passive corruption, because the scheme guaranties that if \textit{t}-servers are corrupted then they will not be able to gain anything.

\item \textbf{Active corruption} happens when the voters (malicious) try to send values that they are not supposed to send - so the voters deviate from the protocol. The PVSS prevent these attacks.
\end{itemize}
