\section{Method}
To ensure a solid systematic approach for this part of the thesis we will use   


\begin{enumerate}
    \item Case
    \item Functional requirements
    \item Quality attribute workshop
        \begin{enumerate}
            \item Quality attributes
            \item Quality attribute scenario
        \end{enumerate}
    \item Architectural decision
       \begin{enumerate}
            \item Tactics and patterns
        \end{enumerate}
    \item Architectural evaluation
\end{enumerate}


\begin{description}
    \item[Case] The purpose of the case is an informal description of the requirements for the electronic voting application. We use this as a introduction to a business/mission for the most important requirements for the electronic voting application.
    
    
    \item[Functional requirements]  The purpose of the functional requirement is to capture the electronic voting application behavior based on the case description. We will take these requirements an categorize them according to the security requirements. Event though the functional requirements isn't in focus in this thesis we will use these requirements helpfull information in the discussion part of the Quality attribute workshop.  
    
    \item[Quality attribute workshop]  The purpose of the Quality Attribute Workshops (QAWs) is a systematic method for identifying a system's architecture critical quality attributes, such as availability, security and modifiability, that are derived from mission or business goals. For the scope of this thesis we will follow the phases in the QAW on a theoretical level, to derive the most important QA for the electronic voting application. We will use the structure of the QAS but we will not hold a practical workshop.  Based on the QA we will formulate the most important Quality attribute scenarios and describe related tactics.
    
    \item[Architectural decision]  This part is about describing the rationale behind the architectural decision. Questions such as what made us end up with this design should be clear. A decision also exclude another decision which should also be reflected in this part. We will use Tyree and Akerman approach for describing our decisions. 
    
    \item[Architectural evaluation]  Evaluating a software architecture is about whether the system meets the quality requirements. Here we will cover how well our architecture meets the quality goals in relation to tradeoffs between quality attributes. We use ATAM analysis for this purpose.  

\end{description}